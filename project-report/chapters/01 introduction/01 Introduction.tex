\chapter{Introduction}\label{cha:introduction}
This report details a programming language’s design, definition, and implementation, as per \gls{aau}'s fourth-semester module description. \cite{AAU_Modules_P4}. The programming language is named Arc for \textit{Arduino Concurrently}, and its purpose is to make it a little simpler to write concurrent programs for the arduino.

\section{Initial problem}\label{sec:initialproblem}
The problem analysis begins from the project proposal: "A Concurrent Programming Language for Arduino." The proposal outlines how microcontrollers, like the Arduino, are used in \gls{cps} to monitor and control the surroundings. However, although sensor input is sent asynchronously to the \gls{cpu}, the Arduino has no direct support for concurrency.

\subsection{Project outline}
The project proposal presents the following tasks and challenges:

\paragraph{Tasks}
\begin{itemize}
    \item Develop a language that integrates Concurrent tasks in Arduino
    \item Give natural support to the \gls{cps} feedback loop
    \item It should have an intuitive domain specific syntax
\end{itemize}

\paragraph{Challenges}
\begin{itemize}
    \item Handle context switch taking \textbf{real-time constraints} in mind
    \item Provide constructs to handle \textbf{race conditions} (atomic sessions, semaphores, asynchronous queues, etc.)
    \item Comply with memory requirements
\end{itemize}

