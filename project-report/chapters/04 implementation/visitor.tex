\section{Visitor pattern}\label{sec:visitorpattern}
This section details working with the integrated visitor pattern that \gls{antlr} provides. There are multiple ways of traversing an \gls{ast}, with the two popular ways being the listener and the visitor patterns. Running the \gls{antlr} tools without any extra flags results in the listener class being chosen by default. The difference between the two is that the listener is called by the walker-object provided by \gls{antlr}, whereas the visitor must be called explicitly to walk the \gls{ast}'s children~\cite{Parr2014}. The listener is great when creating an \gls{ast} in which all of the nodes get visited and none are ignored. It is thus an automatic process.

The visitor design pattern is a way to traverse the \gls{ast} to allow us to add new operations to an existing structure without having to modify it. It makes it both fast and easy to scale classes and fulfills the open/closed principle by allowing extension and no modification. The 'Visitor' interface and the abstract 'BaseVisitor' class created by \gls{antlr} are necessary for the visitor implementation.

An example of how Arc implements a visitor method is seen in Figure~\ref{lst:visitorterminalexpression}. When a terminal expression node is visited in the \gls{ast} traversal, the visitTerminal\_expression method is called by \gls{antlr}. The method declares a string variable 'name' and assigns the context of the terminal expression (given as a formal parameter) as its value, in lines 2-3. It then instantiates a new AST\_node 'terminal,' which is given name as a parameter in line 4. Then it runs through a series of checks in an if-else statement to find out which type the terminal expression should have. \todo{Check that the code matches actual code}


\begin{listing}[htb!]
    \begin{minted}{java}
        public AST_node visitTerminal_expression(arcParser.Terminal_expressionContext ctx) {
        String name;
        name = ctx.getText();
        AST_node terminal = new Terminal_expression_node(name);
        if (ctx.NUMBER() != null)
            terminal.type = Types.NUM;
        else if (ctx.BOOL() != null)
            terminal.type = Types.BOOL;
        else if (ctx.CHAR() != null)
            terminal.type = Types.CHAR;
        else if (ctx.IDENTIFIER() != null) {
            SymbolHashTableEntry id = symbolTable.get(ctx.IDENTIFIER().getText());
            if(id == null){
                throw new RuntimeException("This variable has not been declared '" + ctx.IDENTIFIER().getText() + "'");
            }
            else{
                terminal.type = id.Type;
            }
        } else {
            throw new RuntimeException("problems with terminal expression");
        }
        return terminal;
        }
    \end{minted}
    \caption{Visiting a terminal expression.}
    \label{lst:visitorterminalexpression}
\end{listing}


First (in lines 5-6), it checks if the context contains a number - if so, it assigns the num type to the terminal. Else (in lines 7-8), it checks if the context contains a boolean - if so, it types the terminal to bool. Else (in lines 9-10), it checks if the context contains one or more characters - if so, the terminal type is assigned to char. Else (in lines 11-18), it checks whether or not the context contains an identifier, such as for a function or variable, which is more complex than simply assigning the terminal type immediately.

First, it instantiates a SymbolHashTableEntry 'id' and gets the value from the symbolTable and whatever is stored in the context. It then check whether or not the assignment was successful by checking if the \textit{id} is not null - if it is null, it throws an exception because that means that the variable has not been declared appropriately. Else, it simply assigns the id type to the terminal.

Finally (in lines 19-20), if no checks pass, this means an unexpected error has occurred, and an exception is thrown. This check is purely for debugging purposes and serves no real purpose other than notifying that a problem has occurred in the visitation of the node. The most important thing to notice is that the terminal AST\_node gets assigned a type depending on the if-else statements, and then the method returns the terminal if it does not end, throwing an exception caused by an error.

Another example of an implemented visitor method can be seen in Figure~\ref{lst:visitorplusminusexpression}. In this example, the visitor handles the 'plus' and 'minus' expressions. In lines 2, the method declares an AST\_node 'plus\_minus\_node', similar to what was done in the first example. Then in lines 3-6, it checks if the second child node is a plus operator - if that is the case, it instantiates a new AST\_node and gives it the string "plus" as a parameter. Else, it passes the string "minus" as a parameter.


\begin{listing}[htb!]
    \begin{minted}{java}
    public AST_node visitPlus_minus_expression(arcParser.Plus_minus_expressionContext ctx) {
        AST_node plus_minus_node;
        if (ctx.children.get(1).getText().toCharArray()[0] == '+')
            plus_minus_node = new AST_node("plus");
        else
            plus_minus_node = new AST_node("minus");
        plus_minus_node.child = visit(ctx.expression(0));
        plus_minus_node.child.MakeSiblings(visit(ctx.expression(1)));
        AST_node[] astnodearray = { plus_minus_node.child, plus_minus_node.child.rightSibling };
        if (Typecheck.Check(astnodearray, Types.NUM))
            plus_minus_node.type = Types.NUM;
        else {
            throw new Expression_type_exception("the plus/minus expression has bad typing");
        }
        return plus_minus_node;
    }
    \end{minted}
    \caption{Visiting a plus or minus expression.}
    \label{lst:visitorplusminusexpression}
\end{listing}


Before moving on, the left operand is made the child of the \textit{plus\_minus\_node}, and the right operand a sibling to the left operand child. This happens in lines 7-8. These child nodes are the result of a visit of the children of the ctx\todo{consider ctx in glossary}.

To ensure that the expression is valid and legal in Arc, we instantiate an AST\_node array in line 9, with the two operands as elements, and in lines 10-14 check if they both contain the type Types.NUM - if so, Types.NUM is assigned to the \textit{plus\_minus\_node}s type field. Else, en exception is thrown. Finally, we return the \textit{plus\_minus\_node}.