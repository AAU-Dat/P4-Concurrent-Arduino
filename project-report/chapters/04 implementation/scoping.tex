\section{Scoping}\label{sec:scoping}
Scoping describes when and where a variable is visible, can be referenced, or assigned a value. It is an integral part of any programming language. Identifiers are often reused - some even commonly, such as the iterator 'i' used in loop constructs or 'temp' for temporary local variables. Scoping is very useful for handling the different layers of code. We typically distinguish between two types of scoping: dynamic and static.

Static scoping means that the scope of variables is determined statically before executing a program. Static scoping often has better readability since the code often directly shows the value of a variable. When a reference to a variable is found, and we want to know the value, we can follow the variable from its initialization and look at any changes to the variable's value in previous references. This referencing means that the compiler first looks for the variable in the current block, then the global variables, and then in smaller scopes. Static scoping is used by many languages, including C and C\#.

Although not as popular, dynamic scoping is used in languages such as Perl and Common Lisp. Dynamic scoping works by using the calling sequence of subprograms to determine the value of a variable. It is said to decrease readability while it can be challenging to know precisely which calling sequences of subprograms contain a reference to a variable~\cite{Sebesta2016}. Dynamic scoping, on the contrary, means that the scope of variables is determined at run time. This scoping means that the compiler first looks in the current block, but then it moves on to each successive calling function in reverse. An example of the difference between static and dynamic scoping can be seen in Listing~\ref{lst:scopeexample}.


\begin{listing}[htb!]
    \begin{minted}{c}
        int a;
        void main(){
            a = 10;
            f();
        }

        void f(){
            int a = 20;
            g();
        }

        void g(){
            print (a);
        }
        // static scope, prints: 10
        // dynamic scope, prints: 20
    \end{minted}
    \caption{An example of how static and dynamic scoping differs.}
    \label{lst:scopeexample}
\end{listing}


If the code snippet uses static scoping, the print method will print '10'. While the variable 'a' is not available in the scope of function 'g,' the compiler is going to look in the global scope and find a variable 'a' there. In the 'main' function, the variable 'a' is assigned the value of 10. In function 'f' another variable 'a' is instantiated and immediately assigned the value of 20; however, 'a' in this function refers to the local variable and has nothing to do with the global 'a'.

If the code snippet uses dynamic scoping, the print method is going to print '20'. Instead of looking at the global variables, the compiler goes back to the calling function 'g' and looks for a variable 'a', which it finds and prints.

Arc uses static scoping because it is more readable and easier to understand. C++ uses a variant of static scoping, making it more reasonable for Arc to use a similar scope.

The block structure in our syntax primarily handles the implementation of different scopes. We push and pop the stack of scopes upon visitation. This can be seen in Listing~\ref{lst:VisitBlock}. The visitor begins by pushing the scope onto the stack in line 2; then, it visits its children before popping the scope in line 9 when it is no longer needed.


\begin{listing}[htb!]
    \begin{minted}{java}
    public AST_node visitBlock(arcv2Parser.BlockContext ctx) {
        symbolTable.push();
        AST_node block = new Variable_declaration_node("block");
        List<arcv2Parser.StatementContext> list = ctx.statement();

        for (arcv2Parser.StatementContext statementContext : list) {
        visit(statementContext);   
        }
        symbolTable.pop();
        return block;
    }
    \end{minted}
    \caption{VisitBlock from our visitor.}
    \label{lst:VisitBlock}
\end{listing}


The scoping rules of Arc are implemented everywhere in our semantic visitor. An example of this can be seen in Listing~\ref{lst:ScopeImplementationExample}. When the visitor reaches a for-loop, it calls the symbolTable to check for the identifier given as input and determine if it has been declared correctly. The get function returns null if it does not exist; therefore, we check if the 'entry' identifier is null before continuing.


\begin{listing}[htb!]
    \begin{minted}{java}
    @Override public AST_node visitForloop_statement(arcv2Parser.Forloop_statementContext ctx) { 
        ...
        SymbolHashTableEntry entry = symbolTable.get(ctx.IDENTIFIER(1).getText());
        if (entry == null) {
            throw new RuntimeErrorException(null, "this identifier '" + ctx.IDENTIFIER(1).getText() + "' does not exist" );
        }
        ...
    }
    \end{minted}
    \caption{The SymbolHashTableEntry class}
    \label{lst:ScopeImplementationExample}
\end{listing}