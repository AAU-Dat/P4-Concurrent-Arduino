\chapter{Implementation}\label{cha:implementation}
This chapter covers the implementation of the Arc programming language, which includes a model of the compilation process and the compilation target, a short discussion of the visitor pattern, and a look at the implementation of the visitors, symbol-table, type-checking, and scope-checking.


\section{Compiler}\label{sec:compiler}
The purpose of a compiler is to convert source code into target code. In the case of the Arc compiler, the source code is text written in the Arc language. A generalized model of the compilation process can be seen in Figure~\ref{fig:generalcompilermodel}.

The target code is often low level code for a particular system or runtime, but not always. When the target code is another high level languange, the process is also called transpilation. The Arc languange is used for programming Arduinos, and so the compiler has to either generate Arduino specific machine code, or transpile to the high level Arduino language. Because the Arc language leverages the Protothreads library for the Arduino for its concurrency model, the Arc compiler is a transpiler.


\begin{figure}[htb!]
    \centering
    \includegraphics[width=0.8\textwidth]{figures/Full_Compiler.png}
    \caption{A general model of a compiler~\cite{CraftingCompiler}}
    \label{fig:generalcompilermodel}
\end{figure}


The Arc compiler follows many of the steps of the general model. Figure~\ref{fig:arccompilermodel} models the Arc compiler, with some noteworthy details.

\begin{figure}[htb!]
    \centering
    

    \begin{tikzpicture}[node distance=3cm]


        \begin{scope}[node distance=3cm,local bounding box=clusterA]
            \node (a) [state] {Scanner} node[below,scale=.7, xshift=60,yshift=-20]{Antlr};
            \node (b) [state, right of=a] {Parser};
            \draw[arrow, ->] (a) -- node[above,scale=.70,align=center]{Tokens} (b);
        \end{scope}

        \node(clusterA_g)[cluster,fit=(clusterA)]{};
        \node (start) [shift={($(clusterA.west)+(-2cm,0)$)}] {Source code};
        \node (c) [state, right of = b] {Type checker};
        \node (d) [state, shift={($(c.south)+(0,-1.5cm)$)}] {Scope checker};
        \node (e) [state, shift={($(d.south)+(0,-1.5cm)$)}] {Code generator};
        \node (f) [shift={($(e.south)+(0,-1cm)$)}] {Target code};

        \draw[arrow, ->] (start) -- (clusterA_g);
        \draw[arrow, ->] (b) -- node[right,scale=.70,xshift=-10,yshift=7]{AST} (c);
        \draw[arrow, ->] (c) -- node[right,scale=.70,align=center]{AST} (d);
        \draw[arrow, ->] (d) -- node[right,scale=.70,align=center]{AST} (e);
        \draw[arrow, ->] (e) --  (f);

    \end{tikzpicture}
    \caption{The Arc compiler.}
    \label{fig:arccompilermodel}
\end{figure}

In figure \ref{fig:arccompilermodel}, the source code is first scanned into tokens and then parsed into an AST, Antlr handles all this, therefore the box around scanner and parser. The AST is then type checked. After type check, the code is then scope checked. the type and scope check rules can be read in section 3.4. After the code has been checked further. It is time to code generate the Arc code, into Arduino code. Which is the last step in the compiler. The code is then in Arduino code.  

\section{Lexer and parser generation}\label{sec:lexerandparsergen}

This section briefly covers how the lexer and parser for Arc has been implemented, and what alternatives could have been used.

The lexer, also called a lexical analyzer or scanner, takes a character stream and turns it into a list of tokens. Tokens are a representation of something in a language. A num in Arc would, for example, be a token that represents numbers. The lexer recognizes and can discard characters of the character stream, so that the parser can ignore them. For example is the parser not concerned with whitespaces, and nor does it need to, so it can simply discard them. If the lexer did not discard these, the parser would constantly have to check for them. These tokens are then passed on to the parser, which is tasked to make syntatic sense of them. It compares the tokens and their structure to the grammar of the specific language \cite{Parr2014}.

For Arc, \gls{antlr} has been used to create the grammar, and also to generate both the lexer and the parser. From the written grammar, it is possible for \gls{antlr} to generate the lexer, parser and more \cite{Parr2014}. This made \gls{antlr} highly effective to work with when designing Arc, as minute changes to the grammar could easily be made without a worry - while the generation of the lexer and parser would not have to be recreated with every iteration. Instead, only the grammar would ever need to be updated to reflect our changes, and then \gls{antlr} would be responsible for generating all of the neccessary files based on the new grammar.

The grammar was made in a file called 'arc.g4', which is an \gls{antlr}4 grammar file used to state the rules of the language. We decided to go with a modular approach, and seggregate the lexer and parser rules into separate files. Therefore, the lexer rules can be found in a file called 'lexerRules.g4' and is imported into the main arc grammar file. These two files are everything \gls{antlr} needs to generate all needed files for the parser and the lexer.

Figure \ref{fig:lexerandparserfiles} show the files that \gls{antlr} generates for us when including the '-visitor' flag during compilation. The flag is responsible for generating the parse tree visitor files. The file 'arcLexer.java' is the lexer responsible for the lexical analysis, and the file 'arcParser.java' is the parser. Then there is a list of tokens for both grammar files, and some other boilerplate code, neccessary for the analysis. Finally, the file 'arcVisitor.java' act as the visitor interface, while the file 'arcBaseVisitor.java' is the abstract class implementing the visitor interface. Most of these files are simply boilerplate and thus neccessary for \gls{antlr} to work its magic. Much further detail about these files are beyond the scope of this project.

\begin{figure}[htb!]
    \begin{center}
        \includegraphics[width=0.25\textwidth]{figures/lexerAndParserFiles.png}
        \caption{Files generated for Arcs parser and lexical analyzer, including the flag '-visitor'}
        \label{fig:lexerandparserfiles}
    \end{center}
\end{figure}

Listing \ref{lst:parserandlexerexample} shows how the parser, lexer and visitor is used in practise. In \textbf{line 1} an input is created from a test file. This input is a character stream used by the lexer instantiated in \textbf{line 2}. Then a list of tokens gets created based on the lexical analysis in \textbf{line 3}, which is given to the parser in \textbf{line 4}. Now, that the input has been parsed, the \gls{cst} is created in \textbf{line 5} and is ready to be traversed in \textbf{line 6}, using our custom visitor class inheriting from the 'ArcBaseVisitor' class.

\begin{listing}[htb!]
    \begin{minted}{java}
        CharStream input = CharStreams.fromFileName("src/astTestFile.txt");
        arcLexer lexer = new arcLexer(input);
        CommonTokenStream tokens = new CommonTokenStream(lexer);
        arcParser parser = new arcParser(tokens);
        ParseTree tree = parser.start();
        // Any custom Visitor classes
    \end{minted}
    \caption{An example of how the parser and lexer is used}
    \label{lst:parserandlexerexample}
\end{listing}

We have gained a lot of knowledge about parsers and lexers from the semester courses and by reviewing \gls{antlr}s generated files and documentation; but writing our own from scratch could also have been a high rewarding educational opportunity. Alas, this would also have meant considerably extra development time, while any potential changes also needs to be reflected in all layers of the code. One small change to the grammar file, would also have to be changed in both the lexer and the parser. Using \gls{antlr} to handle that for us, has both saved us time and energy, so that we could focus on learning rather than maintaining a custom solution.
\section{Visitor pattern}\label{sec:visitorpattern}\feedback{We still working on this section, and unsure what is missing}
This section will briefly discuss the two methods ANTLR provides of traversing a \gls{cst}, listeners and visitors, what they are, how they are used, and why they are used in Arc.

The listener pattern, is not responsible for calling methods to traverse the tree. ANTLR generates enter and exit methods for each rule, these methods are then called when the walker encounters a node. The benefits of using the listener pattern, is that it is automatic, and the methods created do not have to explicitly visit all their child nodes\cite{Parr2014}. \feedback{We are unsure of how correct this actually is, taken from: Page 18}

The visitor pattern, is a method used to traverse the \gls{cst} in a specific manner, where methods are used to visit each child node. By using the visitor pattern, there is more control of how the traversal of the \gls{cst} is done, and how much of the \gls{cst} is visited. ANTLR can generate a visitor interface, this interface is created from Arcs grammar, and creates all the needed visitor methods. These act as boilerplate for Arcs needed visitor methods, that can then be written out to do excatly as needed\cite{Parr2014}.

For Arc there is made use of the visitor pattern, because it is not neccecary, for Arc, to visit all nodes, since for some of them it is known what will follow. It is also the more secure option, as it gives the option to be more specific about what happens for each node. This is especially good for futureproofing, as there might be certain things that had not been predicted, and can therefore be specifically fixed.

An example of how Arc implements a visitor method, can be seen in figure \ref{lst:visitorterminalexpression}, this is the visitor method called when visiting a terminal expression node in the \gls{cst}. The method creates a string variable 'name' that is assigned to the value of the method \textbf{.getText} of the formal paramater 'ctx'. From this, an AST\_node variable 'terminal' is made. Then a series of checks are made, to find the type of terminal expression that is being visited, to give 'terminal' the correct type. The first check being, if the formal paramater 'ctx' contains a number, if it does, the type of the terminal is assigned to be NUM, which is Arcs representation of numbers. If it does not contain a number, the next check, which is for boolean, is made, after this the same type of check is made for characters. Then a check for the identifier is made, if it contains an identifer, a SymbolHashTableEntry variable 'id' is made based on the identifer of the variable 'ctx'. If the variable 'id' is null, this means that the variable has not been declared, and an error is thrown, informing the user. If not, then the type of the variable 'terminal' is assigned to the type of the variable 'id'. If ctx\question{maybe another name} does not get caught in any of the checks, it means there is an issue with the terminal that does not fall under any of the considered conditions. When the ctx is caught in a check, and no error is thrown, the type of 'terminal' will be set, and the method returns 'terminal'

\todo[inline]{The description of these functions has to be checked to ensure the understanding of them is correct.}

\begin{listing}[htb!]
    \begin{minted}{java}
        public AST_node visitTerminal_expression(arcv2Parser.Terminal_expressionContext ctx) {
        String name;
        name = ctx.getText();
        AST_node terminal = new Terminal_expression_node(name);
        if (ctx.NUMBER() != null)
            terminal.type = Types.NUM;
        else if (ctx.BOOL() != null)
            terminal.type = Types.BOOL;
        else if (ctx.CHAR() != null)
            terminal.type = Types.CHAR;
        else if (ctx.IDENTIFIER() != null) {
            SymbolHashTableEntry id = symbolTable.get(ctx.IDENTIFIER().getText());
            if(id == null){
                throw new RuntimeException("This variable has not been declared '" + ctx.IDENTIFIER().getText() + "'");
            }
            else{
                terminal.type = id.Type;
            }
        } else {
            throw new RuntimeException("problems with terminal expression");
        }
        return terminal;
        }
    \end{minted}
    \caption{Visiting a terminal expression}
    \label{lst:visitorterminalexpression}
\end{listing}


Another example of how the visitor is used, is seen with figure \ref{lst:visitorplusminusexpression}, that shows how the visitor handles plus and minus expressions. First an AST\_node variable is made, called 'plus\_minus\_node', then a check is made, to determine if the expression is a plus expression, or a minus expression. Depending on the expression type, the variable 'plus\_minus\_node' is set to be a new AST\_node for either plus or minus\question{Unsure if this is correct}. A child node is then assigned to be the first expression of the 'ctx', and a sibling to that child is assigned to the second expression of 'ctx'. With this, the expression has been structured in a way that resebels that of a tree.\question{Maybe just yeet that last part, or go more in depth} An array is made, that contains the first child of the 'plus\_minus\_node' node, and it's right sibling. To ensure that the plus or minus expression is legal in Arc, a check is made on the entire array. This check, makes a type check to ensure that all the child nodes and their siblings are numbers. If this this is true, the type of the 'plus\_minus\_node' node is set to be numer, if not an error is thrown, since Arc does not allow for the plus or minus expression to be used on anything but numbers. Is no error is thrown, the 'plus\_minus\_node' node is returned.

\begin{listing}[htb!]
    \begin{minted}{java}
    public AST_node visitPlus_minus_expression(arcv2Parser.Plus_minus_expressionContext ctx) {
        AST_node plus_minus_node;

        if (ctx.children.get(1).getText().toCharArray()[0] == '+')
            plus_minus_node = new AST_node("plus");
        else
            plus_minus_node = new AST_node("minus");
        plus_minus_node.child = visit(ctx.expression(0));
        plus_minus_node.child.MakeSiblings(visit(ctx.expression(1)));

        AST_node[] astnodearray = { plus_minus_node.child, plus_minus_node.child.rightSibling };
        if (Typecheck.Check(astnodearray, Types.NUM))
            plus_minus_node.type = Types.NUM;
        else {
            throw new Expression_type_exception("the plus/minus expression has bad typing");
        }
        return plus_minus_node;
    }
    \end{minted}
    \caption{Visiting a plus or minus expression}
    \label{lst:visitorplusminusexpression}
\end{listing}


% \begin{listing}
%     \begin{minted}{java}
        
%     \end{minted}
% \end{listing}

%Describe to types of visitors we have used
%Show ALOT of code examples


%Options
%pros and cons of both
%How is it used
%How do we use it


%Vi vil ikke altid ind i alle noder, fordi det ikke er nødvændigt
%Det er sikre valg, da vi kan være specifikke for hver node, hvis der er noget vi ikke har forudset
%Future proof

\section{Scoping}\label{sec:scoping}

Scoping is a description of when and where a variable is visible, can be referenced or assigned a value. It is an important part of any programming language. Identifiers are often reused - some even commonly, such as the iterator 'i' used in loop constructs or 'temp' for temporary local variables and so on. Scoping is very useful for handling the different layers of a code. We typically distinguish between two types of scoping: dynamic and static.

Static scoping, as the name implies, means that the scope of variables are determined statically, before the execution of a program. Static scoping is used by many languages, including C and C\#. Static scoping is said to be more readable, since the code often directly shows the value of a variable. When a reference to a variable is found and we want to know the value, we can follow the variable from its initialisation and look at any changes to the value of the variable in previous references. This means that the compiler first looks for the variable in the current block, then the global variables, and then in smaller scopes.

Dynamic scoping, on the contrary, means that the scope of variables is determined at run time. Dynamic scoping, albeit not as popular, is used in languages such as Perl and Common Lisp. Dynamic scoping works by using the calling sequence of subprograms to determine the value of a variable. It is said to decrease readability while it can be difficult to know exactly which calling sequences of subprograms contains a reference to a variable \cite{Sebesta2016}. This means that the compiler first looks in the current block; but then it moves on to each successive calling function in reverse. An example of the difference between static and dynamic scoping can be seen in Listing \ref{lst:scopeexample}.

\begin{listing}[htb!]
    \begin{minted}{C}
        int a;

        void main()
        {
            a = 10;
            f();
        }

        void f()
        {
            int a = 20;
            g();
        }

        void g()
        {
            print (a);
        }

        // static scope, prints: 10
        // dynamic scope, prints: 20
    \end{minted}
    \caption{An example of how static and dynamic scoping differs.}
    \label{lst:scopeexample}
\end{listing}

If the code snippet uses static scoping, the print method is going to print '10'. While the variable 'a' is not available in the scope of function 'g', the compiler is going to look in the global scope and find a variable 'a' there. In the 'main' function the variable 'a' is assigned the value of 10. In function 'f' another varibale 'a' is instantiated and immediately assigned the value of 20, however, 'a' in this function refers to the local variable and has nothing to do with the global 'a'. If the code snippet uses dynamic scoping, the print method is going to print '20'. Instead of looking at the global variables, the compiler goes back to the calling function 'g' and looks for a variable 'a', which it finds and then prints as the result.

Arc uses static scoping because of it being more readable and easier to understand. C++ uses a variant of static scoping, which also makes it more reasonable for Arc to use a similar scope.

The implementation of different scopes is primarily handled by the block structure in our syntax. We push and pop the stack of scopes upon visitation. This can be seen in Listing {lst:VisitBlock}. The visitor begins by pushing the scope onto the stack in \textbf{line 2}, then it goes on to visit its children before popping the scope in \textbf{line 9}, when it is no longer needed.

\begin{listing}[htb!]
    \begin{minted}{java}
    public AST_node visitBlock(arcv2Parser.BlockContext ctx) {
        symbolTable.push();
        AST_node block = new Variable_declaration_node("block");
        List<arcv2Parser.StatementContext> list = ctx.statement();

        for (arcv2Parser.StatementContext statementContext : list) {
        visit(statementContext);   
        }
        symbolTable.pop();

        return block;
    }
    \end{minted}
    \caption{VisitBlock from our visitor.}
    \label{lst:VisitBlock}
\end{listing}

The scoping rules of Arc is implemented everywhere in our semantic visitor. An example of this can be seen in Listing \ref{lst:ScopeImplementationExample}. When the visitor reaches a for-loop, it makes a call to the symbolTable to check for the identifier given as input and find out if it has been declared properly. The get function returns null if it does not exist, therefore, we check if the 'entry' identifier is null before continuing.

\begin{listing}[htb!]
    \begin{minted}{java}
    @Override public AST_node visitForloop_statement(arcv2Parser.Forloop_statementContext ctx) { 
        ...
        SymbolHashTableEntry entry = symbolTable.get(ctx.IDENTIFIER(1).getText());
        if (entry == null) {
            throw new RuntimeErrorException(null, "this identifier '" + ctx.IDENTIFIER(1).getText() + "' does not exist" );
        }
        ...
    }
    \end{minted}
    \caption{The SymbolHashTableEntry class}
    \label{lst:ScopeImplementationExample}
\end{listing}
\section{Type checker}\label{sec:typechecker}
This section will discuss the type checker for Arc, how it is structured, and present examples of its implementation.

The type checker is responsible for verifying that operands evaluate correctly. It does this by walking through the \gls{ast} from the bottom-up with a visitor pattern~\cite{Parr2014}. If a non permitted type is found, it throws an error along with a message for the developer to fix the issue~\cite{Sebesta2016}.

An example of a type checker throwing an error could be in the case of a developer attempting to use the addition operator on a number and a character. This mistake will throw an error because it is not legal behavior in Arc.

%There are two ways of doing type-checking: statically and dynamically. Static type checking is done at compile-time, while dynamic type checking is done at run time. Usually, it is considered a better practice to use static type checking because it catches any potential errors earlier than a dynamic type checker, and it costs less too~\cite{Sebesta2016}. However, dynamic typing allows for more flexibility when writing code.

%For example, in languages such as Python and JavaScript, variables do not need to have a type on a declaration - all they need is a value. They can also change "type" if the developer chooses to assign a number to a variable containing a string.

%The language Arc transpiles to is C++, a statically typed language. For this reason, Arc will use a static type checker to make the transpilation more one-to-one. Although, an argument could be made for Arc to use dynamic type-checking since dynamically typed languages tend to be faster for writing code. Since time (cost) is a limiting factor for our target group, it would make sense to make a language that encourages less time spent writing code. However, Arc is only made to work with simple concurrency, and time is better spent elsewhere.

An example of type checking in Arc can be found in Listing~\ref{lst:typecheckerexample}. The example shows the part of the type-checker that handles assignments. First, in lines 1-2, a SymbolHashTableEntry 'entry' is created for scope checking, and an \gls{ast} node 'expression' is created for the type checking. The expression node is created by visiting the expression of the context received as a formal parameter.


\begin{listing}[htb!]
    \begin{minted}[label=Typechecking in SemanticVisitor.visitAssignmentStatement]{java}
        SymbolHashTableEntry entry = symbolTable.get(ctx.IDENTIFIER().getText());
        AST_node expression = visit(ctx.expression(0));

        if (entry == null) {
            throw new RuntimeException(" this varible '" + ctx.IDENTIFIER().getText() + "' does not exist");
        } else if (entry.Mutability != true) {
            throw new RuntimeException(" this varible '" + ctx.IDENTIFIER().getText() + "' is not mutable and therefore cannot be assigned to");
        } else if (entry.Type != expression.type) {
            throw new Expression_type_exception("the assingment expression has bad typing");
        } else{
            symbolTable.insert(entry);
        }
    \end{minted}
    \caption{Code snippet of type checking in Arc.}
    \label{lst:typecheckerexample}
\end{listing}


The first check on lines 4-5 is part of the scope-checking and has already been covered. On lines 6-7, the visitor checks whether or not the entry is mutable (meaning that it is allowed to be modified). If this is not the case, an error will be thrown. Else in lines 8-9, we check if the entry has the expected type from the 'expression' \gls{ast} node created earlier. If there has been a type mismatch, the compiler throws an error. Else, we insert the entry in the symbol table in line 11, and the entry has been successfully assigned.
\section{Code Generation}

how is it done? through visitor pattern and antlr generated ast


This section will explain the code gen part of the compiler. In this part the target
code is generated. The target code is in this case the arduino languege is essentially c++ with some arduino constructs. The Code is generated using another visitor pattern quite simelar to the one we use for semantic analysis. It follows the divide and conquer principle and uses the antlr genrerated ast. It then genrates the code going through the ast and passing the generated code between the nodes using our CodeGenStringObject.

\begin{listing}[htb!]
    \begin{minted}{java}
        public class CodeGenStringObject {
            public String GlobalScope = "";
    public String Setup = "";
    public String Loop = "";

    public String Type_Coverter (String input){
        
        if(input.equals("mut num") || input.equals("num" )){
            input = "float";
        }
        
        if(input.equals("mut char")){
            input = "char";
        }

        if(input.equals("mut bool")){
            input = "bool";
        }
        
        return input;
    }
    
}
\end{minted}
\caption{code gen object used in code gen}
\label{lst:code gen object}
\end{listing}

we use the CodeGenStringObject because a node sometimes will generate code in 3 different places of the final code at once. Hence we have a made a object containing 3 different strings and a method. The 3 string each corespond one of the 3 areas that can be generated to, GlobalScope, the Setup structure and the loop structure. This makes it possible to do all code gen simply in a single visit of the 3 beacuse its not necessary to keep track of 3 different placements in the file and switch between writing in them. The final code gen is done by using the final CodeGenStringObject and simply inserting the 3 strings into the file in order. \todo{this needs to be more clear and use an example to show it} The method in CodeGenStringObject is simply used in code gen to rewrite the types that we have in the input to the corresponding types in c++.







benefits of using antlr ast




improtance of keeping the same order as the written program

how it works with example
    -tasks
    \begin{listing}[htb!]
        \begin{minted}{java}
    @Override
    public CodeGenStringObject visitTask_declaration(arcv2Parser.Task_declarationContext ctx) {
        CodeGenStringObject cpp = new CodeGenStringObject();

        String ptName = "pt" + Integer.toString(get_task_number());
        String ptNameThread = ptName + "thread";

        cpp.GlobalScope += "pt " + ptName;
        cpp.GlobalScope += ";\n";
        cpp.GlobalScope += "int " + ptNameThread + "(struct pt *pt) { \n PT_BEGIN(pt);\n for(;;){ \n";

        // region Every

        if (ctx.EVERY() != null) {
            List<StatementContext> list = ctx.statement();
            for (StatementContext statement : list) {
                cpp.GlobalScope += visit(statement).GlobalScope;
            }
            cpp.GlobalScope += "\nPT_SLEEP(pt, " + ctx.NUMBER().getText() + ");\n";
            cpp.GlobalScope += "\n}\n";
        }

        // endregion

        // region When

        else if (ctx.WHEN() != null) {
            cpp.GlobalScope += "if (" + visit(ctx.expression()).GlobalScope + ") { \n";

            List<StatementContext> list = ctx.statement();
            for (StatementContext statement : list) {
                cpp.GlobalScope += visit(statement).GlobalScope;
            }
            cpp.GlobalScope += "\n}\n";
        }

        // endregion

        // region Default

        else {
            List<StatementContext> list = ctx.statement();
            for (StatementContext statement : list) {
                cpp.GlobalScope += visit(statement).GlobalScope;
            }
            cpp.GlobalScope += "\n}\n";
        }

        // endregion

        cpp.GlobalScope += "PT_END(pt);\n}";

        cpp.Setup += "PT_INIT(&" + ptName + ");";
        cpp.Loop += "PT_SCHEDULE(" + ptNameThread + "(&" + ptName + "));";

        return cpp;
    }

    private int count = 0;

    public int get_task_number() {
        int temp = count;
        count += 1;
        return temp;
    }

    }
        \end{minted}
        \caption{code showing how task are generated}
        \label{lst:code gen task}
    \end{listing}



    -for look
    -pin declaration


% \section{MOVE ME EVENTUALLY}
% This section contains text that has to be moved to proper sections.

% \subsection*{Arc implemementation of the sample project}
% The sample project implemented in Arc.

% \begin{listing}[htb!]
%     \begin{minted}[label=Arc example]{text}
%         #pin BUTTON_PIN(12, OUTPUT);
%         mut num buttonState = 0;
%         mut bool ledState = HIGH;

%         task(bool ledState) every 1000 {
%             ledState = not ledState;
%             digitalWrite(LED_BUILTIN, ledState)
%         }

%         task(num buttonState) {
%             buttonState = digitalRead(BUTTON_PIN);
%             Serial.Println(buttonState);
%         }
%     \end{minted}
%     \caption{Project example implemented in Arc, assuming print is possible.}
%     \label{lst:arcexample}
% \end{listing}
