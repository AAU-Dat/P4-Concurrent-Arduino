\section{Code Generation}

how is it done? through visitor pattern and antlr generated ast


visitor pattern

benefits of using antlr ast

code gen object


\begin{listing}[htb!]
    \begin{minted}{java}
public class CodeGenStringObject {
    public String GlobalScope = "";
    public String Setup = "";
    public String Loop = "";

    public String Type_Coverter (String input){
        
        if(input.equals("mut num") || input.equals("num" )){
            input = "float";
        }

        if(input.equals("mut char")){
            input = "char";
        }

        if(input.equals("mut bool")){
            input = "bool";
        }

        return input;
    }

}
    \end{minted}
    \caption{code gen object used in code gen}
    \label{lst:code gen object}
\end{listing}


improtance of keeping the same order as the written program

how it works with example
    -tasks
    \begin{listing}[htb!]
        \begin{minted}{java}
    @Override
    public CodeGenStringObject visitTask_declaration(arcv2Parser.Task_declarationContext ctx) {
        CodeGenStringObject cpp = new CodeGenStringObject();

        String ptName = "pt" + Integer.toString(get_task_number());
        String ptNameThread = ptName + "thread";

        cpp.GlobalScope += "pt " + ptName;
        cpp.GlobalScope += ";\n";
        cpp.GlobalScope += "int " + ptNameThread + "(struct pt *pt) { \n PT_BEGIN(pt);\n for(;;){ \n";

        // region Every

        if (ctx.EVERY() != null) {
            List<StatementContext> list = ctx.statement();
            for (StatementContext statement : list) {
                cpp.GlobalScope += visit(statement).GlobalScope;
            }
            cpp.GlobalScope += "\nPT_SLEEP(pt, " + ctx.NUMBER().getText() + ");\n";
            cpp.GlobalScope += "\n}\n";
        }

        // endregion

        // region When

        else if (ctx.WHEN() != null) {
            cpp.GlobalScope += "if (" + visit(ctx.expression()).GlobalScope + ") { \n";

            List<StatementContext> list = ctx.statement();
            for (StatementContext statement : list) {
                cpp.GlobalScope += visit(statement).GlobalScope;
            }
            cpp.GlobalScope += "\n}\n";
        }

        // endregion

        // region Default

        else {
            List<StatementContext> list = ctx.statement();
            for (StatementContext statement : list) {
                cpp.GlobalScope += visit(statement).GlobalScope;
            }
            cpp.GlobalScope += "\n}\n";
        }

        // endregion

        cpp.GlobalScope += "PT_END(pt);\n}";

        cpp.Setup += "PT_INIT(&" + ptName + ");";
        cpp.Loop += "PT_SCHEDULE(" + ptNameThread + "(&" + ptName + "));";

        return cpp;
    }

    private int count = 0;

    public int get_task_number() {
        int temp = count;
        count += 1;
        return temp;
    }

    }
        \end{minted}
        \caption{code showing how task are generated}
        \label{lst:code gen task}
    \end{listing}



    -for look
    -pin declaration