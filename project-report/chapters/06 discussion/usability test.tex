\section{Usability test}\label{sec:discussion_usabilityTest}

This section will cover discussions about the usability tests of Arc, how much of a difference it could have made and most importantly how much of an impact it could have had on the design of Arc. Since we didn't have time to conduct a usability test, it is difficult to surmise any meaningful conclusions about any potential results. We have theorised tests for getting feedback on the readability and writeability of Arc in Section~\ref{subsec:usabilityTestOfArc}. These tests would have provided us with an insight and a better foundation for planning future improvements, based on how the participants interacted with the programming language.

The readability test would have the participants read through some prepared examples of Arc programs and explain with their own words what is going on in the code. The participants would be divided into either one of two groups: beginners and experienced programmers. This has been considered a simple test internally, and would be meaningful way to document how well programmers of various experience understands the code in its intended form. We could have considered other tests for readability, such as showing them some snippets of the syntax and thereby extract information about their opinion on the syntax directly. However, this would only have complicated both the test itself and the following analysis hereof, unnecessarily. It would likely make it difficult or impossible for the unexperienced programmers to follow along in a meaningful way. The integrity of their responses would also be compromised, by having the syntax explained to them before making their own assumptions of the meaning. Therefore, we believe that we have come up with the better test originally, and that would be the test to conduct, had we had the time to do so.

The writability test would have been very similar to the readability test; but instead of reading prepared programs, we would have had them write their own simple programs using the Arc language to their best ability, after having familiarised themselves with the general idea through the readability test. However, this might prove to be difficult for some of the beginner programmers, while they might not understand the task or have difficulty in writing code on the spot in a new language. In case the test participant has no experience at all, it would seemingly be impossible to complete such a test without more instructions and time to learn some fundamental coding ethics. This could turn out to be more of a test in being able to understand a given assignment and creating a solution program in a new language, more than it is a meaningful test of Arc's readability.

Calculating the cost of Arc is contingent on the test results. One thing we could have done to help users learn Arc and thereby reduce the potential cost, would be to include a documentation of the language, rather than just examples. This would help developers who prefer to read through the documentation when learning a new language. While Arc is a relatively simple language, this would not even have to be very long, compared to how much beneficial it might prove to be.

Usability tests need much consideration and preparatory work before it can be useful for any project. While Arc only has theoretical tests, we can only theorise about any improvements or results. If we had more time, we would have done more work on the usability test and conducted as many as we could to further enhance the capabilities and ease of use of Arc.