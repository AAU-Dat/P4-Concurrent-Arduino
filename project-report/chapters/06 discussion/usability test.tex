%we could have had a usability test
%how would have the tests have helped us designing Arc
%expected resuts could differ from actual results
%time spent on analyzing results
%Should more focus have been put on developing tests
%

\section{Usability test}
This section will discuss the usability tests of Arc, what could have been done differently and what impact it would have had on Arc. Since no usability tests were conducted on Arc, it is not easy to anticipate what the results would mean for the development and design. Currently, only speculation of how to test could be conducted been made; for now, tests for readability and writeability of Arc have been theorized. These tests would give insight into how actual users interact with the syntax of Arc, where it is lacking, and what should be considered for improvement. 

The readability test consists of users being given some Arc code and for them to explain what is happening. The users would be categorized by their previous experience in coding. This was thought to be a simple test that would give insight into what users of different levels of experience are easily understood. Other ways of testing readability could have been considered, such as showing some syntax of Arc, explaining what it means, and asking a test participant if the syntax makes sense and what they might want to change. This way of testing might give a more in-depth analysis of the readability but could also be biased as the syntax would be explained before making their assumptions of the meaning. This way of testing the readability would also complicate the analysis of the results, as there is no clear yes or no answer to the readability of the example, as there is with the current way of testing. Of course, there are many different ways of doing this type of test, but for the context of Arc, the current way of testing is deemed suitable for the wanted results.

Writability tests of Arc were made similar to the readability tests. Participants are given a task they must create based on what they saw in the readability test. This test seemed like a good option for testing the writeability of Arc, but many things should be considered for this. Some participants might not understand what the task they are to do must do. Also, for people who have never coded before, this test might not test Arc's writeability but more a test of their ability to understand the assignment and create a programmatic solution. 

Calculating the cost of Arc would change depending on what type of data the readability and writability tests return. One thing that might help some users, and reduce costs, would be to include documentation instead of only examples. This could help guide some users that prefer reading through the documentation to learn a new programming language. 

Usability tests need much consideration, and since Arc only has theoretical usability tests, it is not easy to have a proper discussion about their outcome and what could have been done differently. 
Although usability tests could have been a good idea, the time needed for creating, arranging, and analyzing them, far exceeds what was available. For this reason, it was deemed appropriate only to make use of theoretical usability tests that can then be made use of in the future.



