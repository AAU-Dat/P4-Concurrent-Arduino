\section{Language design}

This section will discuss if and how well, Arc has lived up to the language criteria made in section \ref{sec:languageeval} and also how well Arc has lived up to the learning goals for the semester.

Arc is, a language made to introduce concurrency to hobbyists. To quantify the requirements, a table \ref{tab:priorityofcharacteristics} was made of how important the characteristics of the language were.
Simplicity was very important for Arc, as the language is made with hobbyists in mind. Many things influence simplicity, and keeping a language that works with concurrency simple has proven difficult, as concurrency is a complex topic. Therefore it was important for the concurrency structures especially, to be easy to use and understand. Arc consists of three concurrency structures, \textbf{When}, \textbf{Where}, and the idle task. These structures make use of words that would generally be used to describe when something should happen, apart from the idle task, which is the most abstract of the task variants. Because the structures are made to resemble spoken language to describe when a task is meant to be called, it increases the simplicity of the language, and should also increase readability as the tasks are called, in the same manner as they are read. 



