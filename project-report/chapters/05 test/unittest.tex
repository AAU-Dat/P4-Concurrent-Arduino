\section{Unit test}\label{subsec:unittest}
Unit testing is a test method that checks whether specific code does what is expected of it. A unit test is a breakdown of functionality of code into descrite testable units~\cite{UnitTestBasic}.

Unit tests in this project will follow the AAA structure, which deivdes tests into three parts. This specific structure can be seen in the Listing~\ref{lst:unit test with aaa structure}.

\begin{listing}[htb!]
    \begin{minted}[label=Unit test with AAA structure]{java}
        @Test
        public void nameOfTest(){
            // Arrange is the part where we initialize objects and sets the value of the data that is passed to the method under test.

            // Act is the part where it invokes the method under test with the arranged parameters.

            // Assert is the part that verifies the action of the method under test behaves as expected. 
        }
    \end{minted}
    \caption{The AAA structure}
    \label{lst:unit test with aaa structure}
\end{listing}

Constructing unit tests with this structure, ensures unit tests follow the same format and course of action. An example has been created to show what we mean is a great and usable unit test, and this can be seen in the Listing~\ref{lst:unit test of make sibling}.

\begin{listing}[htb!]
    \begin{minted}[label=Unit test of make sibling]{java}
        @Test
        public void MakeSiblings_TestFunction_ReturnTrue(){
            // Arrange
            AST_node a = new AST_node();
            AST_node b = new AST_node();
            AST_node c = new AST_node();
            a.adoptChildren(b);
            b.adoptChildren(c);
            // Act
            b.MakeSiblings(c);
            // Assert
            assertEquals(a, b.parent);
            assertEquals(a, c.parent);
            assertEquals(b, a.child.leftMostSibling);
            assertEquals(c, a.child.rightSibling);
            assertEquals(c, b.rightSibling);
            assertTrue(c.rightSibling == null);
            assertEquals(a.child.leftMostSibling, c.leftMostSibling);
        }
    \end{minted}
    \caption{Unit test of make sibling}
    \label{lst:unit test of make sibling}
\end{listing}

This specific unit test tests the MakeSibling() function, where we test the individual scenario to make sure that when a node gets a new sibling all of the data is updated correctly. We run the MakeSibling() on node B and then check whether all of the data of node C gets updated to the expected values. If the test is passed, we know that the MakeSibling() does what is expected. 

Unit tests ensure that the code works the same after refactoring. All unit tests can be rerun if anything is broken. If something is broken, the test fails and can be corrected.