\subsection{Usability Test of Arc}\label{subsec:usabilityTestOfArc}
In this section, the usability test of Arc will be described. However as there was not enough time to conduct the test, this section will only describe how the test was intended to be conducted.

Before continuing we will shortly mention what a usability test is and which results you can expect from conducting a usability test. A usability test is a tool that programmers use to identify any usability problems and collect quantitative- and qualitative data to dertimine the overall satisfaction with the product or service. In overall a usability test can be refered to as a evaluation of a product or service to the representive users.

The purpose of a usability test on the Arc language is to determine how Arc performs compared to other languages/libraries when it comes to how easy it is to write concurrency. Therefore, Arc will be evaluated on the ease of use to write concurrency and not how fast it executes or it's memory usages or other performance realted metrics.


\subsubsection{Aim of the test}\label{subsubsec:aimOfTheTest}
The aim of the usability test is to dertimine how Arc performs compared to other programming language/libraries and to see if the Arc language have lived up to our expectations which can be seen in [ref: table 3.2]. The metrics the performance will be based on, is how easy Arc is to use. The metrics in mind here is those criteria from Language evaluation criteria which can be seen at [ref: table 3.1]. As mentioned in [ref: table 3.2] we have for example: focused more on simplicity and syntax design rather than support for abstraction. But, this test will still be covering all of the charateristics to make sure that Arc lives up to the expectations and is easy to use for new programmers. The reason why we have arranged the charateristics in [ref: table 3.2] they way we have is because Arc was created to ease the use on how to write concurrency.

The most important things to test in this usability test is to cover all of the different charateristics to show that the Arc language lives up to the expectations it intentionally was created to.

The following questions that must be covered when this test has been conducted
\begin{enumerate}
    \item How easy is Arc to get familiar with and get up to speed with?
    \item What do the new programmer think of Arc?
    \item What do more experiend programmers think of Arc compared to another programming language?
    \item How would the experiend programmer evaluate Arc using the Language Evaluate Criteria?
\end{enumerate}

\subsubsection{The test participants}\label{subsubsec:theTestParticipants}
For this usability test, the participants need to have different programming skills so they can be divided into two groups each presenting the novice programmer and experiend programmer. The novice programmer is needed to determine whether or not Arc would be a language that is great to start with. This will give an indication on the ease of use of Arc.

The experiend programmers would be needed to compare Arc to other programming languages. As well as their knowlegde or usage of different programming language will be great to determine which part of Arc that would work well and which parts that would not works so well. It would also be helpfull to get the experiend programmer to evaluate Arc by using the Language Evaluate Criteria which we can compare to our own to determine if our language lived up to the expectations.

For the usability test, it is always best to have as many participants as possiple to give the best results. But in this case, three to four participants to each of the novice- and experiend programmers should be enough to give us sufficient data wihtin the time frame of this semester.

\subsubsection{The test session}\label{subsubsec:theTestSession}



\subsubsection{Task list}\label{subsubsec:taskList}

%Make some task the participants should had go thourh with the think loud method
\begin{enumerate}
    \item
\end{enumerate}

\subsubsection{Analysis of the data}\label{subsubsec:analysisOfTheData}
    

\subsubsection{Summary of usability test of Arc}



%Language evaluation criteria:
%(x) marks the things that should be tested for in the usability test (add those that are nessecary):

%1. (x) simplicity
%2. Orthogonality
%3. Data types
%4. (x) Syntax design
%5. Support for abstraction
%6. Expressivity
%7. Type checking
%8. Exception handling
%9. Restricted aliasing.