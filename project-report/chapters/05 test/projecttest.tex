\section{Usability Test of Arc}\label{subsec:usabilityTestOfArc}
A usability test is a valuable tool for projects to identify usability challenges and collect both quantitative- and qualitative data which can be used to conclusions aobut the Arc Language
determine the overall user satisfaction with the project's product.

A usability test on the Arc language would help determine how Arc compares to existing libraries and languages when handling concurrency on an Arduino. Therefore, Arc would be evaluated on its ability to handle concurrency and the writability of the language rather than how fast it executes, its memory usages, or other performance-related metrics.

The usability test aims to determine how the Arc language measures up to our preformed expectations, as seen in~\ref{tab:priorityofcharacteristics}. Arc would be evaluated by the Language Evaluation Criteria, as seen in~\ref{tab:langevalcrit}. The focus of this project has been chiefly on simplicity and syntax design rather than support for abstraction. However, this test will still cover all of the characteristics to ensure that Arc lives up to the expectations and is easy to use for hobbyist developers. We have arranged the characteristics in~\ref{tab:priorityofcharacteristics},in such a way that it should make it easier to use concurrency on an Arduino.

The hypothesis we will test with this usability test is 

\blockquote{Arc is easy to learn and lives up to the criteria from Tables~\ref{tab:priorityofcharacteristics}\ref{tab:langevalcrit}}

We have four questions that we would want answered to test the hypothesis: How easy is it to get familiar with and use Arc? What does a beginning programmer think of Arc? What does a more experienced programmer think of Arc, and how does it compare to other programming languages? How would an evaluation of Arc based on the Language Evaluation Criteria by the experienced developer look?

\subsection{The test participants}\label{subsubsec:theTestParticipants}
For this usability test, we would need two groups of participants; the control group would perform the test in the Arduino language, and a test group to perform the test in the Arc language.

It is typically best to have as many participants as possible for a usability test to provide the best possible results. However, we would only need a handful of participants in each group to have satisfactory results for this test.

\subsection{The test}\label{subsubsec:theTestSession}
There would be three parts: the reading test, code test, and questionnaire. The reading test would work by showing test participants snippets of code, and the participant would describe what they expect the code does. The writing part would work on giving the participant an exercise to try and write in the Arc/Arduino code, depending on their group. The questionnaire would be questions about their experience with the test itself and the programming language. 

\subsubsection{The reading test}
The reading test is designed to evaluate the readability of Arc compared to Arduino. Three small code examples are described in Arc and Arduino to achieve this. The examples are the same for both languages and should include at least a concurrency example. 

The test participants then have to explain the code examples for their participant group qualitatively. Additionally, the test can be timed to obtain quantitative data. 

\subsubsection{The writing test}
After the reading test, the participants have a writing test. This test is designed to evaluate the writability of Arc versus Arduino. 

The writing test consists of three small implementation tasks, one of which must be concurrency-related. This test must include the project blueprints and a good explanation of the desired behavior. Additionally, good documentation for both languages must be presented for a fair comparison. 

This test should primarily supply quantitative data: degree of correctness, measured by the number of mistakes, the time of completion, and a ratio between these, representing the cost criteria. The size of the source code file might also be relevant. 

Examples for the code tasks in this test:
\begin{enumerate}
    \item Toggles an LED on and off every second.
    \item Toggles an LED on and off when a condition has been met, such as a button press.
    \item Reads the state of a button as often as it can, unconditionally and without a sense of time.
\end{enumerate}

\subsubsection{The questionare}
Finally, after the reading and writing test, participants have to answer a short questionnaire, following the questionnaire chapter in~\cite[p.~146,~224-235]{benyon_2014}.

Examples of qualitative questions we would like answered: 

\begin{enumerate}
    \item How would you rate the Arc language as a beginner friendly programming language?
    \item How does it compare to any other programming language you might know of?
    \item How easy did you find the task function to work with when constructing new threads?
    \item Would you recommend other developers to use Arc, why or why not?
    \item How did the Arc language compare to your first programming language, with focus on simplicity?
    \item How does it compare to any other programming languages you have used in the past?
    \item Do you have any prior experience with concurrency and/or the Arduino?
    \item How would you evaluate the Arc language based on these Language Evaluation Criteria?
\end{enumerate}

More quantatitive questions: 
\begin{enumerate}
    \item On a scale from 1-10 how easy did you find the reading tasks?
    \item On a scale from 1-10 how easy did you find the writing tasks?
    \item On a scale from 1-10 how likely are to recommend the language to a friend?
\end{enumerate}

\subsection{Data analysis}\label{subsubsec:analysisOfTheData}
After conducting all the tests, we would have to analyze all the gathered data and compare the results. For example, if one of the participants was stuck during a task - it could indicate that the participant had an issue with something; but if a multitude of the participants had a similar issue - it would indicate that something might be innately wrong with either the language or the task itself.

The reading test would be analyzed by comparing the ratio of correct to inaccurate descriptions and potentially the level of detail in the answers compared between languages. 

The writing test data is primarily quantitative and can be plotted and compared directly. It also needs a reasonably large sample size to be statistically sound. If enough data points are obtained, this data will show us if there is a difference in writability between Arc and the Arduino language, and if any, maybe even how large.  

The final thing to analyze would be the replies to the questionnaire. In particular, we are interested in the differences in answers between the control group and the test group: is there a difference in the perceived difficulty between the languages, especially considering the measured differences, if any.
