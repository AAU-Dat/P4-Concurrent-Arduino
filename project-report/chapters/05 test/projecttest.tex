\section{Usability Test of Arc}\label{subsec:usabilityTestOfArc}
In this section, the usability test of Arc will be described as a hypothetic test. However, as there was not enough time to conduct the test within the timeframe of this semester, this section will only describe how the test was intended to be conducted.

Before continuing, we will shortly mention what a usability test is and which results can be expected from conducting a usability test. A usability test is a tool that programmers use to identify usability problems and collect quantitative- and qualitative data to determine the overall satisfaction with the product or service. Overall, a usability test can evaluate a product or service to the representative users.

A usability test on the Arc language aims to determine how Arc performs compared to other languages/libraries when it comes to how easy it is to write concurrency. Therefore, Arc will be evaluated on the ease of use to write concurrency and not how fast it executes its memory usages, or other performance-related metrics.


\subsection{Aim of the test}\label{subsubsec:aimOfTheTest}
The usability test aims to determine how Arc performs compared to other programming languages/libraries and to see if the Arc language has lived up to our expectations, as seen in [ref: table 3.2]. The metrics the performance will be based on are how easy Arc is to use. The metrics in mind here are those criteria from the Language evaluation criteria, which can be seen in [ref: table 3.1]. As mentioned in [ref: table 3.2], we have focused more on simplicity and syntax design rather than support for abstraction. However, this test will still cover all of the characteristics to ensure that Arc lives up to the expectations and is easy to use for new programmers. We have arranged the characteristics in [ref: table 3.2] the way we have, because Arc was created to ease the use of how to write concurrency.

The most important thing to test in this usability test is to cover all the different characteristics to show that the Arc language lives up to its intended expectations.

The following questions that must be covered when this test has been conducted
\begin{enumerate}
    \item How easy is Arc to get familiar with and get up to speed with?
    \item What does the new programmer think of Arc?
    \item What do more experienced programmers think of Arc compared to another programming language?
    \item How would the experienced programmer evaluate Arc using the Language Evaluate Criteria?
\end{enumerate}

\subsection{The test participants}\label{subsubsec:theTestParticipants}
For this usability test, the participants need different programming skills to be divided into two groups, each presenting the novice and experienced programmers. The novice programmer is needed to determine whether or not Arc would be a great language to start with. The data from both the novice- and experienced programmers will indicate the ease of use of Arc.

The experienced programmers would be needed to compare Arc to other programming languages. As well as their knowledge or usage of the different programming languages will also be excellent for determining which part of Arc would work well and which parts would not work so well. It would also be helpful to get the experienced programmer to evaluate Arc using the Language Evaluate Criteria, which we can compare to our own to determine if our language lived up to the expectations.

It is always best to have as many participants as possible for the usability test to give the best results. Nevertheless, in this case, three to four participants for each novice- and experienced programmer should be enough to give us sufficient data within the time frame of this semester.

\subsection{The test session}\label{subsubsec:theTestSession}
For each of the test participants, a test session will be conducted. If the participant allows it, the session will be recorded to capture the participant's reactions and actions. A test moderator will instruct the participant on what to do and help them if necessary for each test session. When conducting the test sessions, the moderator must be impartial and not influence the test results.

During a test session, the test moderator will tell the participant where the test will be conducted and which task they will attempt to complete during the test session. The test moderator needs to tell the participant to think aloud during the test session since the data will be analyzed and used to help evaluate the Arc language. If the participant does not have any questions, the test may begin. A session should not take more than 15 to 30 minutes, depending on the participant and how many questions they have asked asked during the session. Besides the test moderator, there will be a test observer who takes notes of how and what the test participant did.

When a test participant is conducting the test, the test moderator must not interrupt the participant, as it is essential to get accurate impressions of their actions and reactions of the Arc language. If the participant stops thinking aloud, the test moderator must remind them to do so since collecting this kind of data is crucial. If the participant does not have any questions, the moderator will ask questions to both the novice- and experienced programmers. The question for the novice programmers could be:
\begin{enumerate}
    \item What do you think of Arc as a starting programming language?
    \item Furthermore, how was it to Arc instead of your everyday language?
    \item Did the task() function make it easier to construct new threads?
    \item Would you recommend Arc to other people as their first language?
\end{enumerate}

The questions for the experienced programmers could be like:
\begin{enumerate}
    \item With focus on the simplicity of Arc, how did it compare to your first programming language?
    \item Which things worked well? And if something did not work for you, what was it?
    \item How would you evaluate the Arc language based on the Language Evaluation Criteria?
\end{enumerate}

\subsection{Task list}\label{subsubsec:taskList}
The following task would be the task the novice- and experienced programmers would have to solve to help give us data to determine if Arc would live up to the criteria.
\begin{enumerate} % I dont exactly know which task could be nice to do put these is my examples ¯\_(ツ)_/¯
    \item Read a task() that turns a led on and off every one second with your own words to see if it is easy to read
    \item Read a task() that turns a led on based on a condition with your own words to see if it is easy to read
    \item Read a task() that turns a led on and off with no time delay of condition with your own words to see if it is easy to read
    \item Create a task() that turns a led on and off every one second
    \item Create a task() that turns a led on based on a condition
    \item Create a task() that turns a led on and off with no time delay of condition
\end{enumerate}

\subsection{Analysis of the data}\label{subsubsec:analysisOfTheData}
After the test sessions have been conducted, all the data gathered from the test sessions would have to be analyzed. However, as there was no time this semester's timeframe to conduct the test, this section will only describe what we would have been analyzing and looking into.

After conducting the test sessions, the first thing that needs to be done is to go through the test observer's notes and the recorded material to check what led the observer to take the notes one did. An example could be: If one of the participants got stuck on a task and said something about it being confusing to read the task() with the conditional. The test observer's notes should match the problem, so later on, this problem can be taken in to evaluating the Arc language. If multiple test participants had the same problem where they had a hard time continuing, this would have to be reviewed more extensively, as this could potentially mean that something is not as it is supposed to be.

The next thing that should be spent time analyzing is the test participant's answers to the question asked throughout the test session. These answers should contain vital information about how Arc performed and what it was like to use Arc as a programming language. A question that will be especially important to analyze is what the experienced programmer answered to how they would evaluate the Arc language based on the Language Evaluate Criteria. The answers would have to be compared to the other answers and our expectations for the Arc language for differences and similarities.

Another thing worth comparing is the time difference between how quickly the novice and the experienced programmer finished the task. If the time difference is not significant, it might indicate that the Arc language is simple and easy to read and write, which means that the language lives up to the expected criteria. As all participants have been asked the same number of questions, it will not be unfair to compare the time difference. Keep in mind that comparing the time difference would be insufficient if the novice programmers got more help than the experienced programmers. 

Time is just one thing that can be compared. Another thing like the number of questions and how well they formulated the questions could also be analyzed and compared, as this would also indicate how easy or hard the language was to use for each of the groups of programmers.