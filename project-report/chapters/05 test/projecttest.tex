\subsection{Usability Test of Arc}\label{subsec:usabilityTestOfArc}
%Plan for creating project test:
%0. Tell shortly that the section would be about usability testing of Arc but it will only be described as there was not enough time to conduct it.
In this section, the usability testing of Arc will be described. However as there was not enough time to conduct the test, this section will only describe how the test was intended to be conducted. So the test described is hypothetical.

%1. What is a usability test and its criteria? In which cases does a usability come ind handy?
Before continuing we will shortly mention what a usability test is and which results you can expect from conducting a usability test. Usability test is a tool that programmers use to identify any usability problems and collect quantitative- and qualitative data to dertimine the overall satisfaction with the product. In overall a usability test can be refered to as a evaluation of a product or service to the representive users.

%2. What is the purpose of a usability test and what do you gain from conducting the test?
The purpose of a usability test on the Arc language is to determine how Arc performs compared to other languages/libraries when it comes to how easy it is to write concurrency. Therefore, Arc will be evaluated on the ease of use to write concurrency and not how fast it executes or it's memory usages or other performance realted metrics.

%3. How would you have conducted the test and which thing have you chosen to test from the criteria?
\subsubsection{Aim of the test}\label{subsubsec:aimOfTheTest}
The aim of the usability test is to dertimine how Arc performs compared to other programming/libraries. The metrics the performance will be based on is how easy it is to used. So the metrics in mind here is those criteria that meassure how good a language can be. The reason for these performance metrics is that Arc was created to ease the use on how to write concurrency.

The most important thing to test is how well it performs on the measured metric compared to other language. To acheive this we think that including newly programmers aswell as experienced programmers will give us the best data that we then can conlude if our language performance better or not.

The following question must be covered under the test:
\begin{enumerate}
    \item How easy is Arc to get familiar with?
    \item What do the newly programmers think about Arc compared to their first programming language?
    \item What do the experienced programmers think about Arc compared to other programming language with focus on making concurrency easier to write?
    \item What do the experienced programmers think about Arc compared to other language with focus on complexity?
    \item How would experienced programmers evaluate Arc using the Language Evaluation Criteria?
\end{enumerate}

\subsubsection{The test participants}\label{subsubsec:theTestParticipants}
\subsubsection{The test session}\label{subsubsec:theTestSession}
\subsubsection{Task list}\label{subsubsec:taskList}
\subsubsection{Analysis of the data}\label{subsubsec:analysisOfTheData}
    
%4. Summary
%    4.1 Its hard to evalute on something if the test is hypothetical but else 4.2 and 4.
%    4.2 How well went the testing
%    4.3 Did we get the expected results
\subsubsection{Summary of usability test of Arc}



%Language evaluation criteria:
%(x) marks the things that should be tested for in the usability test (add those that are nessecary):

%1. (x) simplicity
%2. Orthogonality
%3. Data types
%4. (x) Syntax design
%5. Support for abstraction
%6. Expressivity
%7. Type checking
%8. Exception handling
%9. Restricted aliasing.