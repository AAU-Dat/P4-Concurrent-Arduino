\chapter{Conclusion}\label{cha:conclusion}

This chapter will conclude the project and evaluate on Arc. This includes how well the learning objectives of the semester has been met during the project and whether or not Arc is a successful solution to the problem statement.

% \section{Learning Objectives}\label{sec:learning_objectives}
% This semester project aimed to "learn how to design and implement a programming language and how this process can be supported by formal definitions of language syntax and semantics and techniques and methods of translator - and/or interpreting - construction".

% A set of learning objectives has been stated in our module description. This serves as a list of criteria to accurately measure the success of our work. We are expected to have obtained the following knowledge through our work on the project. We have selected some of these that we believe showcase that we have succeeded with the project. 

%\textbf{Understand and explain the basic concepts in a formal definition of the syntax and semantics of a programming language.} We believe that we have accomplished this objective and have shown our understanding of the concepts as we have explained the entire process from conception to implementation, this can be seen in chapter~\ref*{cha:languagedesign}.

%\textbf{Document knowledge and overview of the techniques and concepts involved in language design and translation design.} We have utilized many techniques, and concepts learned in the courses throughout the semester. Especially the Languages and Compilers course has been beneficial when designing the Arc programming language.

\textbf{Explain the individual phases and the correlation between the phases in a translator/interpreter.} We have attempted to explain the phases as a whole, but we are unsure whether or not we have satisfactorily explained the more delicate details. This is primarily due to \gls{antlr} handling most of it for us. We have looked into what the tool does for us and compared it to the knowledge we have obtained through the semester work and found that we understand the individual processes, regardless of how much focus the report has on it.

%\textbf{Describe the implementation techniques used in the constructed translator/interpreter.} We have explained most of the ideas behind our implementation. While many visitors are very similar, we have found examples to present the overarching techniques used to implement our design in our language.

%\textbf{Use correct terminology.} We have used terminology appropriately throughout the entire report and believe that the extent of our new vocabulary is satisfactory.


\textbf{Describe the syntax and semantics of a programming language by using appropriate methods for formal definition.} We have described our syntax and semantics primarily by the terminology, techniques and methods learned in the Syntax and Semantics course. Some techniques have also been presented in the Languages and Compilers course, but this has been chiefly about language design.

\textbf{Implement a translator or interpreter into a specific programming language or an extension to an existing programming language.} We have implemented a transpiler to translate Arc into Arduino. Our internal tests have worked as intended, so we are confident that we have accomplished this.

\textbf{Test the implemented translator or interpreter at all levels: unit, integration and acceptance testing.} We have not implemented a full testing suite for the compiler. However, we have developed a testing strategy for unit, integration, and acceptance tests in the shape of our useability test. Therefore we feel we have partially succeeded in achieving this learning goal.

%\textbf{Account for configuration management during the development of translator or interpreter.} We have accounted for how we use \gls{antlr} and how we have gone about developing our solution, from reading the documentation to researching and trying out different tools and methods throughout the process.

%\textbf{Reason data logically on and with the concepts and techniques involved.} We have used our computer science knowledge to reason when making choices and discussing our project in the group and with our supervisor. We have tried choosing the right tools for the job.

\textbf{Assess the use and usability of known tools and techniques for the definition and implementation of programming languages.} We believe that we have accomplished this in section~\ref{cha:problemanalysis}.

%\textbf{Understand and explain how concrete linguistic concepts are represented at driving times and in formal semantics.} We have accomplished to show this to a certain degree. While we have not gone into much detail about the different concrete linguistic concepts, we have used them in our work.

\todo{Choose five learning objectives and include them here -> then delete the learning goals file.}

The problem statement Arc set out to solve, is as follows: 
\textbf{The project seeks to create a programming language for hobbyists that satisfies the language criterion created, Which is able to utilize concurency to easily express a concurrent flow of control concisely.}

As a product from solving the problem statement we created our language, Arc. The syntax of Arc mostly resembles C; but with minor changes, such as the for-loop which is influenced by the way Python handles it, and the task construct created to handle concurrency. The task construct and any other decision in the project has been driven by the wish to make Arc as easy and intuitive as possible for our target group to write concurrent code for the Arduino. Arc gets translated into C++ and implements Protothreads, to work concurrently on the Arduino.

To ensure that Arc work as intended, a concurrent program was written in C++ with the Protothreads library. The program was rather simple and included two \gls{led}s, the first one was tasked to turn on an off every second, while the other diode was tasked to toggle the light state by a button press. We then recreated the same program using Arc and did not run into any problems.

\question{did we re-create it, or did we simply translate it to the same program?}

In order to definetily conclude that the syntax of Arc concisely expresses a concurrent flow of control, we would have needed to test the usability of the language. From our perspective and limited internal testing, we can surmise a conclusion. The concurrency structures of Arc is colloquial in nature, and should therefore be somewhat intuitive to setup and understand for any hobbyist using our language.

Thereby, we can conclude that Arc is an acceptable solution to the initial problem statement. There is obviously always room for improvement, and we have learned some lessons throughout the project that would make for a better language next time. Testing is very neccessary in the future.