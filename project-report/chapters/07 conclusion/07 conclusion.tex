\chapter{Conclusion}\label{cha:conclusion}

This chapter will conclude the project and evaluate on Arc. This includes how well the learning objectives of the semester has been met during the project and whether or not Arc is a successful solution to the problem statement.

The overall learning goal for this semester was to implement a programming language and understand the underlaying processes. The result of the compiler, is that it is a programming language and forfills the goal on making a programming language. The learning goals for one the underlaying processes was to "Understand and explain the basic concepts in a formal definition of the syntax and semantics of a programming language ". This goal has been forfilled, by using the theories and knowledge from the Syntax and Semantix course in this semester, this is shown in the report by dedicating chapter \ref{cha:languagedesign} on how the language syntax and semantics works in the language Arc. Therefore it would be concluded as the group understands the conceps of syntax and semantics.

The second learning goal was to "Document knowledge and overview of the techniques and concepts involved in language design and translation design;". This goal has been forfilled on how the thought processes on making the Arc programming language. This can be seen in section \ref{sec:languageeval}, on why we choose a programming language to be simple and easy to learn for hobbists. Much of the knowledge for this learning goal comes from the Language and Compiler course from this semester. 
 
The third learning goal was to "Explain the individual phases and the correlation between the phases in a translator/interpreter;". Much of the knowledge for different phases in a compiler comes from the Language and Compiler course from this semester as it describes the phases in detail. The phases of the compiler in the Arc language is explained in section \ref{sec:compiler}, the learning goal is helped to understand what needed to be in the compiler, when planning on the implementation of the compiler for the Arc language. 
 
The forth learning was to "Describe the implementation techniques used in the constructed translator/interpreter;". The knowledge comes from the Language and compiler course, as it the different techniques, this helped the knowledge on what the benefits are with the chosen technique visitor pattern which Arc uses. Visitor pattern is described in section \ref{sec:visitorpattern}.  

\todo{Choose five learning objectives and include them here -> then delete the learning goals file.}

The problem statement Arc set out to solve, is as follows: 
\textbf{The project seeks to create a programming language for hobbyists that satisfies the language criterion created, Which is able to utilize concurency to easily express a concurrent flow of control concisely.}

As a product from solving the problem statement we created our language, Arc. The syntax of Arc mostly resembles C; but with minor changes, such as the for-loop which is influenced by the way Python handles it, and the task construct created to handle concurrency. The task construct and any other decision in the project has been driven by the wish to make Arc as easy and intuitive as possible for our target group to write concurrent code for the Arduino. Arc gets translated into C++ and implements Protothreads, to work concurrently on the Arduino.

To ensure that Arc work as intended, a concurrent program was written in C++ with the Protothreads library. The program was rather simple and included two \gls{led}s, the first one was tasked to turn on an off every second, while the other diode was tasked to toggle the light state by a button press. We then recreated the same program using Arc and did not run into any problems.

\question{did we re-create it, or did we simply translate it to the same program?}

In order to definetily conclude that the syntax of Arc concisely expresses a concurrent flow of control, we would have needed to test the usability of the language. From our perspective and limited internal testing, we can surmise a conclusion. The concurrency structures of Arc is colloquial in nature, and should therefore be somewhat intuitive to setup and understand for any hobbyist using our language.

Thereby, we can conclude that Arc is an acceptable solution to the initial problem statement. There is obviously always room for improvement, and we have learned some lessons throughout the project that would make for a better language next time. Testing is very neccessary in the future.