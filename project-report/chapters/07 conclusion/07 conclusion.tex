\chapter{Conclusion}\label{cha:conclusion}

During this project, we have designed a programming language, Arc, with a language specification containing a description of the syntax, contextual analysis, and semantics. Furthermore, the compiler has been implemented in Java with \gls{antlr} generating the lexer and parser. We have also prepared unit and integration test strategies.

The problem statement Arc set out to solve, is as follows: 

\blockquote{The project aims to create a programming language for hobbyists that satisfies our language criteria and makes it possible to easily express concurrency on the Arduino.}

We can conclude that Arc partly solves the problem statement, as we have a functioning compiler and language that can express concurrency. We can only partly conclude that the problem statement is solved. We cannot guarantee that hobbyists can express concurrency more easily than in the standard Arduino language without a completed usability test. We can also not make any conclusions about the language criteria without conducting the described usability test in section~\ref{subsec:usabilityTestOfArc}.

Overall, the project was successful, even though we did not completely solve the problem statement. We have achieved the learning objectives of the semester. We will show this by going through a curated list of the learning goals, accompanied by a description of how we have achieved them.

\textbf{Explain the individual phases and the correlation between the phases in a translator/interpreter.} We have created a model of the Arc compiler based on a general model for compilers, which reflects our understanding of the compiling process.

\textbf{Describe the syntax and semantics of a programming language by using appropriate methods for formal definition.} We have described our syntax and semantics primarily by the terminology, techniques and methods learned during the project and described in our language design in chapter~\ref{cha:languagedesign}.

\textbf{Implement a translator or interpreter into a specific programming language or an extension to an existing programming language.} We have implemented a transpiler to translate Arc into the Arduino language. It is implemented in java according to the language specification and ready for further development.

\textbf{Test the implemented translator or interpreter at all levels: unit, integration and acceptance testing.} We have not implemented a complete testing suite for the compiler. However, we have developed a testing strategy for unit, integration, and acceptance tests in the shape of our useability test.

\textbf{Assess the use and usability of known tools and techniques for the definition and implementation of programming languages.} We believe that we have accomplished this in chapter~\ref{cha:languagedesign} in discussing the different parser generators.

A concurrent program was written in the Arduino language with the Protothreads library to ensure Arc worked as intended. The program was relatively simple and included two \gls{led}s; the first one was tasked to turn on and off every second, while the other \gls{led} was tasked to toggle the light state by a button press. We created a similar program using Arc, which worked as expected and did not run into any problems.

We can conclude that Arc is a success in regards to the whole semester as it achieves the learning goals and can compile, although it does not fulfill the problem statement completely. There is always room for improvement, and we have learned lessons throughout the project that would make for a better language in the future.