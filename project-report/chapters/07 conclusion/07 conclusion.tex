\chapter{Conclusion}\label{cha:conclusion}

During this project, we have designed a programming language, Arc, with a language specification containing a description of the syntax, contextual analysis, and semantics. Furthermore, the compiler has been implemented in Java with \gls{antlr} generating the lexer and parser. We have also prepared unit and integration test strategies.

The problem statement Arc set out to solve, is as follows: 
\textbf{The project aims to create a programming language for hobbyists that satisfies our language criterion and makes it possible to easily express concurrency on the Arduino.}

We can conclude that Arc partly solves the problem statement, as we have a funcitoning compiler and language, that is able to express concurrency. The reason we can only partly conclude that the problem statement is solved, is that without a usability test, we cannot guarantee that hobbyists can express concurrency more easily than in the stadard Arduino language. We can also not make any conclusions about the language criterion without having conducted, the described usability test~\ref{subsec:usabilityTestOfArc}.

As a whole, we believe that the project is a success, eventhough we did not completely solve the problem statement. That is because, we belive that we have achived the learning objectives of the project. We will show this through going a curated list of the learning goals, accompanied by a description of how we have achived them.

\textbf{Explain the individual phases and the correlation between the phases in a translator/interpreter.} We have attempted to explain the phases as a whole, but we are unsure whether or not we have satisfactorily explained the more delicate details. This is primarily due to \gls{antlr} handling most of it for us. We have looked into what the tool does for us and compared it to the knowledge we have obtained through the semester work and found that we understand the individual processes, regardless of how much focus the report has on it.

\textbf{Describe the syntax and semantics of a programming language by using appropriate methods for formal definition.} We have described our syntax and semantics primarily by the terminology, techniques and methods learned in the Syntax and Semantics course. Some techniques have also been presented in the Languages and Compilers course, but this has been chiefly about language design.

\textbf{Implement a translator or interpreter into a specific programming language or an extension to an existing programming language.} We have implemented a transpiler to translate Arc into Arduino. Our internal tests have worked as intended, so we are confident that we have accomplished this.

\textbf{Test the implemented translator or interpreter at all levels: unit, integration and acceptance testing.} We have not implemented a full testing suite for the compiler. However, we have developed a testing strategy for unit, integration, and acceptance tests in the shape of our useability test. Therefore we feel we have partially succeeded in achieving this learning goal.

\textbf{Assess the use and usability of known tools and techniques for the definition and implementation of programming languages.} We believe that we have accomplished this in section~\ref{cha:problemanalysis}.

A concurrent program was written in the Arduino language with the Protothreads library to ensure that Arc worked as intended. The program was relatively simple and included two \gls{led}s; the first one was tasked to turn on and off every second, while the other diode was tasked to toggle the light state by a button press. We then created a simillar program using Arc, which worked as expected and did not run into any problems.

We can conclude that Arc is a succes in regards to the whole semester as it archieves the learning goals and is able to compile, eventhough it does not fullfill the problem statement completely. There is always room for improvement, and we have learned lessons throughout the project that would make for a better language in the future. 