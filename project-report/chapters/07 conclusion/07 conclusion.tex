\chapter{Conclusion}\label{cha:conclusion}

During this project, we have designed a programming language, Arc, with a language specification containing a description of the syntax, contextual analysis, and semantics. Furthermore, the compiler has been implemented in Java with \gls{antlr} generating the lexer and parser. We have also prepared unit and integration test strategies.

The problem statement Arc set out to solve, is as follows: 
\textbf{The project aims to create a programming language for hobbyists that satisfies our language criterion and makes it possible to easily express concurrency on the Arduino.}

We can conclude that Arc partly solves the problem statement, as we have a funcitoning compiler and language, that is able to express concurrency. The reason we can only partly conclude that the problem statement is solved, is that without a usability test, we cannot guarantee that hobbyists can express concurrency more easily than in the stadard Arduino language. We can also not make any conclusions about the language criterion without having conducted, the described usability test~\ref{subsec:usabilityTestOfArc}.

As a whole, we believe that the project is a success, eventhough we did not completely solve the problem statement. That is because, we belive that we have achived the learning objectives of the project. We will show this through going a curated list of the learning goals, accompanied by a description of how we have achived them.

The overall learning goal for this semester was to implement a programming language and understand the underlaying processes. The result of the compiler, is that it is a programming language and forfills the goal on making a programming language. The learning goals for one the underlaying processes was to "Understand and explain the basic concepts in a formal definition of the syntax and semantics of a programming language ". This goal has been forfilled, by using the theories and knowledge from the Syntax and Semantix course in this semester, this is shown in the report by dedicating chapter \ref{cha:languagedesign} on how the language syntax and semantics works in the language Arc. Therefore it would be concluded as the group understands the conceps of syntax and semantics.

The second learning goal was to "Document knowledge and overview of the techniques and concepts involved in language design and translation design;". This goal has been forfilled on how the thought processes on making the Arc programming language. This can be seen in section \ref{sec:languageeval}, on why we choose a programming language to be simple and easy to learn for hobbists. Much of the knowledge for this learning goal comes from the Language and Compiler course from this semester. 
 
The third learning goal was to "Explain the individual phases and the correlation between the phases in a translator/interpreter;". Much of the knowledge for different phases in a compiler comes from the Language and Compiler course from this semester as it describes the phases in detail. The phases of the compiler in the Arc language is explained in section \ref{sec:compiler}, the learning goal is helped to understand what needed to be in the compiler, when planning on the implementation of the compiler for the Arc language. 
 
The forth learning was to "Describe the implementation techniques used in the constructed translator/interpreter;". The knowledge comes from the Language and compiler course, as it the different techniques, this helped the knowledge on what the benefits are with the chosen technique visitor pattern which Arc uses. Visitor pattern is described in section \ref{sec:visitorpattern}.  

%As a product of solving the problem statement, we created our language, Arc. The syntax of Arc mostly resembles C, but with minor changes, such as the for-loop, which is influenced by the way Python handles it, and the task construct created to handle concurrency. The task construct and any other decision in the project have been driven by the wish to make Arc as easy and intuitive as possible for our target group to write concurrent code for the Arduino. Arc gets translated into C++ and implements Protothreads to work concurrently on the Arduino.

A concurrent program was written in the Arduino language with the Protothreads library to ensure that Arc worked as intended. The program was relatively simple and included two \gls{led}s; the first one was tasked to turn on and off every second, while the other diode was tasked to toggle the light state by a button press. We then created a simillar program using Arc, which worked as expected and did not run into any problems.


%The concurrency structures of Arc are colloquial and should therefore be somewhat intuitive to set up and understand for any hobbyist using our language. In order to conclude that the syntax of Arc concisely expresses a concurrent flow of control, we would have needed to test the usability of the language. From our perspective and limited internal testing, we can surmise a conclusion.

We can conclude that Arc is a succes in regards to the whole semester as it archieves the learning goals and is able to compile, eventhough it does not fullfill the problem statement completely. There is always room for improvement, and we have learned lessons throughout the project that would make for a better language in the future. 