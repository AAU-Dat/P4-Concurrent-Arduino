\section{Problem statement}

The problem statement Arc set out to solve, is as follows: 
\textbf{To create a programming language for the Arduino with a set of concurrency constructs, that leverages the Protothreads library, to concisely express a concurrent flow of control.}
To solve this problem statement, Arc was made. The syntax of Arc resembles that of C, but with a few changes that were deemed better suited for hobbyists to use, such as the num type, that replaces all numbering types in C. Effort was especially put into the concurrency structures of Arc, since they were the main solution to the problme Arc set out to solve. These structures are Arcs tasks, that use trigger words to describe when they are executed. The trigger words were used, in an effort to make it clear to users, how each task functions, and to concisely express the flow of the code. Arc could then be translated to Protothreads, so that it could be used on the Arduino.

To ensure that Arc works as expected, a program was first written in Protothreads that made use of concurrency. The program was simple, it was two LEDs, one blinking every second and one that turns on when a button is pressed. The program was then re-created in Arc, and it could then be concluded that Arc works with its concurrency structures. \question{did we re-create it, or did we simply translate it to the same program?}

To conclude, that Arcs syntax concisely expresses a concurrent flow of control, usability tests would be needed, but since they have not been done, it can not be guaranteed. To conclude this part of the problem statement, only the developers perspective is currently avaliable, which will be partly biased. Arcs concurrency structures, use a resemblance of spoken language, to explain when they are to be executed. We believe this approach has led to a concise way to express the flow of control, and can therefore partly conclude that the concurrency structures concisely express a concurent flow of control. \question{Maybe this is too much repetition}

From this, it can be concluded that Arc has solved the problem statement it set out to solve. Ofcourse there are areas that could be improved upon and use further testing, but we belive that Arc solves the problem statement, to a satifying degree. \question{maybe mention future works?}


