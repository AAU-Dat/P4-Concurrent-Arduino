\section{Learning Objectives}\label{sec:learning_objectives}

This semester project aimed to "learn how to design and implement a programming language and how this process can be supported by formal definitions of language syntax and semantics and techniques and methods of translator - and/or interpreting - construction".

A set of learning objectives has been stated in our module description. This serves as a list of criteria to accurately measure the success of our work. We are expected to have obtained the following knowledge through our work on the project.

\textbf{Understand and explain the basic concepts in a formal definition of the syntax and semantics of a programming language.} We believe that we have accomplished this objective and have shown our understanding of the concepts as we have explained the entire process from conception to implementation.

\textbf{Document knowledge and overview of the techniques and concepts involved in language design and translation design.} We have utilized many techniques, and concepts learned in the courses throughout the semester. Especially the Languages and Compilers course has been beneficial when designing the Arc programming language.

\textbf{Explain the individual phases and the correlation between the phases in a translator/interpreter.} We have attempted to explain the phases as a whole, but we are unsure whether or not we have satisfactorily explained the more delicate details. This is primarily due to \gls{antlr} handling most of it for us. We have looked into what the tool does for us and compared it to the knowledge we have obtained through the course work and found that we understand the individual processes, regardless of how much focus the report has on it.

\textbf{Describe the implementation techniques used in the constructed translator/interpreter.} We have explained most of the ideas behind our implementation. While many of the visitors are very similar and such, we have found examples to present the overarching techniques used to implement our design in our language. We might have abstracted the process too much, but we hope that our underlying knowledge shows.

\textbf{Use correct terminology.} We have used terminology appropriately throughout the entire report and believe that the extent of our new vocabulary is satisfactory.

We are expected to have developed the following skills through our work on the project:

\textbf{Describe the syntax and semantics of a programming language by using appropriate methods for formal definition.} We have described our syntax and semantics primarily by the terminology, techniques and methods learned in the Syntax and Semantics course. Some techniques have also been presented in the Languages and Compilers course, but this has been chiefly about language design.

\textbf{Implement a translator or interpreter into a specific programming language or an extension to an existing programming language.} We have implemented a transpiler to translate Arc into C++ so that our languages can be compiled into something that runs on an Arduino. Our internal tests have worked as intended, so we are confident that we have accomplished this.

\textbf{Test the implemented translator or interpreter at all levels: unit, integration and acceptance testing.} Unfortunately, we did not have the time to complete our testing plan. We all agree that a greater focus should be put on testing. However, we have hypothesized a usability test for both readability and writeability. Both reflected and discussed how this would and could have influenced Arc's design and implementation.

\textbf{Account for configuration management during the development of translator or interpreter.} We have accounted for how we use \gls{antlr} and how we have gone about developing our solution, from reading the documentation to researching and trying out different tools and methods throughout the process.

\textbf{Reason data logically on and with the concepts and techniques involved.} We have used our computer science knowledge to reason when making choices and discussing our project in the group and with our supervisor. We have tried choosing the right tools for the job.

We are expected to have optained the following competences through our work on the project:

\textbf{Assess the use and usability of known tools and techniques for the definition and implementation of programming languages.} We believe that we have accomplished this in the report.

\textbf{Understand and explain how concrete linguistic concepts are represented at driving times and in formal semantics.} We have accomplished to show this to a certain degree. While we have not gone into much detail about the different concrete linguistic concepts, we have used them in our work.