The overall learning goal for this semester was to implement a programming language and understand the underlaying processes. The result of the compiler, is that it is a programming language and forfills the goal on making a programming language. The learning goals for one the underlaying processes was to "Understand and explain the basic concepts in a formal definition of the syntax and semantics of a programming language ". This goal has been forfilled, by using the theories and knowledge from the Syntax and Semantix course in this semester, this is shown in the report by dedicating the language design chapter \todo{ref to chapter 3}on how the language syntax and semantics works in the language Arc. Therefore it would be concluded as the group understands the conceps of syntax and semantics.

The second learning goal was to "Document knowledge and overview of the techniques and concepts involved in language design and translation design;". This goal has been forfilled on how the thought processes on making the Arc programming language. This can be seen in section \todo{ref to language criteria}, on why we choose a programming language to be simple and easy to learn for hobbists. Much of the knowledge for this learning goal comes from the Language and Compiler course from this semester. 
 
The third learning goal was to "Explain the individual phases and the correlation between the phases in a translator/interpreter;". Much of the knowledge for different phases in a compiler comes from the Language and Compiler course from this semester as it describes the phases in detail. The phases of the compiler in the Arc language is explained in section \todo{ref compiler}, the learning goal is helped to understand what needed to be in the compiler, when planning on the implementation of the compiler for the Arc language. 
 
The forth learning was to "Describe the implementation techniques used in the constructed translator/interpreter;". This can be read in section \todo{visitor pattern}  
 
Use correct terminology
