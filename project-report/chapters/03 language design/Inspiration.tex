\section{Inspiration to our language}\label{sec:inspiration}
When designing a programming language it is a good idea to look at other programming languages for inspiration. We began by discussing the languages we knew: C, JavaScript, and C\#. It was also important to check C++, as it was the language, that would be used as the taget language of our transpiler. During the group discussion, there was found what was liked in each language and the reason why it was liked. From this, there could be made decisions about what was to be implemented in Arc. More modern languages were also looked at for inspiration: Dart, Kotlin, Rust, Python, and Zyg. 

By doing this, many ideas were taken from many different programming languages, some were modified to fit our needs or wants better and others were taken straight from one language and put into ours. Therefore there will be seen inspiration from many programming languages, when describing Arc.


\subsection{Types}\label{sec:types}
When designing a language, the amount of types and what types to be implemented is an important decisions, to guide this decision there were some set criteria for the language that it must live up to. The language must be a simple language, able to work concurrently on the Arduino, be simple to use for hobbyists, and also the language has a time limit for completion, which is also to be taken into consideration. We have a presumption that fewer types in a language, to a certain degree, increases the simplicity of the language. According to \todo{Ref to criteria tabel from Sebesta}, simplicity in a language increases readability, writeability and reliability. Since fewer types give users less types to consider when coding, the choice should become more clear, of what to use. Having a lot of types has the benefit of giving users more options and being able to work on more specific tasks easier.

For our criteria, many types might not be necessary. For a small and simple language that can give hobbyists the opportunity to work with concurrency, a handfull of types will be sufficient. Also, fewer types will help reduce time spent on language design, and give more time to focus on the concurrency aspects of the language. Therefore, in Arc there will only be three types for now: Num, Bool, and Char. Num for numbers that can work with arithmetic, Bool for boolean, to evaluate expressions and give either true or false values, and Char for some simple string manipulation\question{Is it called string manipulation?}. These types should be sufficient enough, for hobbyists to create some simple concurrent code, thereby living up the criteria we set for the language. For needs above what is in Arc, other languages or libaraies must be used. 

\paragraph*{Num}
Num, is for all numbers in Arc. The name num, is simply used to make it more recognizable for users. Languages such as C, have many different types for different categories of numbers, integers, floats, doubles, and so on. Other languages such as JavaScript only have two types to represent numbers, Number and BigInt, where number can store both integers and floats.
The choice of including many different and specified types for numbers, as in languages such as C, gives users more specified control over the code that they write. This gives users many options and can help in specific situations, where a very specific type is the optimal choice.
Compared to the option of using few or a single type to represent numbers in Arc, would, in our presumption, lead to a more simple language, since many things are abstracted away from the user. Information that might not be relevant for users doing simple arithmetic.

For Arc, a basic ability to manipulate numbers simply, will be sufficient. For this, a direction in a similar manner to how JavaScript handles numbers, will be used. Since the goal of Arc, is simple concurrency, in-depth control of arithmetic will not be a concern, and focus will be on giving users an easy way to begin work with concurrency.

\paragraph*{Bool}
Bool is included in Arc, as in many other languages, and simply evaluates to either true for false. The value of a bool is written in quotation marks, as "true" and "false", unlike a language such as C, where the value could be 0 or 1. We assume this makes it more readable and easy to understand for hobbyists. Boolean is an important type to implement, not only for some of Arcs concurrency structures, but for code in general, as it is used to evaluate expressions. 

\paragraph*{Char}
Char is for all characters in Arc. There are a few different ways of giving users the ability to manipulate text. C does not have built-in strings, but instead uses character arrays. Other languages such as Python, have built-in strings. Arc will use char in a similar way to C, where char is a single character, and can be put into an array, to make a character array. Since for simple concurrency, basic string manipulation will be sufficient.


\subsection{Control Structures}
As mentioned, there has been taken inspiration from languages such as C, C\#, Python, JavaScript, and others. There are three main types of control structure: sequence, conditional, and iteration.\todo{Måske en kilde?} These are structures such as if statements, for loop, while loops, and so on. Since many languages use these common control structures, we have decided to do the same.

\paragraph*{If statements}
The if statement is a very common conditional structure. The syntax for it is similar to how many other languages structure it, with a keyword 'if', some statements and the option to add an 'else'. The reason the structure has not been changed, is because Arc is aimed at hobbyists who might me new to coding. Therefore there has chosen a similar syntax to many other languages, such as C, C\# or JavaScript, so that in the future users may have an easier time learning new languages.

\paragraph*{For loop}
In the same way that the structure of the if statement was not changed compared to other languages, much has not been changed about the iteration structure, 'for' loop either. It is written with a keyword\todo{Do we call it keyword, or terminal?} 'for', then declaring a variable, another keyword 'in', with another variable which could be an array. This is the followed by some statements.
This structure is made to resemble that of Python, where a lot of the work of iterating through something is done behind the scene.

\paragraph*{While loop}
The iterative structure, 'while' loop is, as the if statement and for loop, simmilar to how other widely used programming languages use it. With a keyword 'while', with an expression in paratheses, that when evaluated to true will execute the body. The body is the set of statements in brackets. 

\paragraph*{Switch case}
The switch case structure has been omitted from this language. It stated off as being called 'when' and to be used as many other languages structure the switch case. But by discussing the design of the language, the when structure became less favored, and it was therefore decided to simple ommit it from the language to also simplify the creation of the language itself.
