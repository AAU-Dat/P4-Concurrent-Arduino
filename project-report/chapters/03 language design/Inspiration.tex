\section{Inspiration to our language}\label{sec:inspiration}
When designing a programming language it is a good idea to look at other programming languages for inspiration. We started out, by discussing the languages that we all had worked with in previouse semesters, which were C, JavaScript and C\#. It was also important to check C++, as it was the language, our language was going to compile to.

During the group discussion, we all tried to find things that other programming languages did well, what we each liked in a language and why we liked it. This was done to get an idea for what our programming language should include.
When discussing the syntax of the languages we looked at, we agreed that many of these languages might not be very intuitive for a hobbyist. We therefore needed to do more research on how other languages did this.

We started looking at more modern languages, to gather ideas from them, and see what we could do to make our language simpler for hobbyists, to learn and write. The languages Dart, Kotlin, Rust, Python and Zyg were all looked at, for inspiration. One thing we all looked for in the languages, was if they had a readable syntax and what made it readable.

By doing this, many ideas were taken from many different programming languages, some were modified to fit our needs or wants better, others were taken straight from one language and put into ours. Therefore you will see inspiration from many programming languages, when describing ours.\urgent{A lot of we, you and ours}



%indsæt den del af grammeren der er af types her
%hvorfor så få
%lille sprog, tid* osv.
%fokus på concurrency
%bool er important - event baseret
%char - fordi man skal bruge characters til LCD'er (no print, do we need it?)
%Num - behandles kun som floats
%Mafs can be fucky wucky
%Simple to have a single category of numbers, not seperate 
%Nothing is mutable by default
%Harder to fuck stuff up in concurrency
%Types must be relevant for concurrency and other parts of our criteria
\subsection{Types}
Types implemented in Arc are limited to only be relevant for the criteria set for it. This helps users, since only relevant types are implemented, thereby not overwhelmimg them. By limiting users options, this increases readability, since there are not many different types of variables to keep in mind when reading through code. This limitation is also a benefit for designing the language, as this ensure that focus is on the criteria set, concurrency for Arduino.
Arc is made for basic concurrency, for this hobbyists will not need complex data types or structures. They need some numbers to manipulate, booleans to compare and evaluate expressions and some characters, incase they wish to print something to a screen of some sort. For any needs above this, the hobbyist should be proficient enough at coding, to find other languages or libaraies to support their needs.

Num is for all numbers in the language, and using a name that is just short for number should make it recognizable for most users. It was decided to simply treat all numbers as floats, This would keep all numbers as one type, and not split into many different categories. The goal with this is to simplify the language for beginners and hobbyists. As mentioned before, limiting the amount of information a user must keep track of and not overwhelmimg them. Other languages such as C, has many different types of numbers, int, float, short and so on, different types of numbers for different usecases. Since there has been taken inspiration from languages such as C, it might have made sense to do the same, but for the criteria set for Arc, there was no need for many different types of numbers, only some way to manipulate numbers.
An option could be argued for, instead of one type num, there could have simply been made use of int and float, for whole numbers and decimals. This would not have hindered readability by much, and might make some things more clear for users who wish to use other languages in the future.

Bool is like the bool in other languages, as it can only be true or false. We saw no reason to change this, and therefore decided to keep it as many other languages. The value of a bool is written in quoteation marks, the reason for it to be written like this, and not with a syntax such as C, where the value could be 0 or 1, is because it is more readable and easy to understand.The reason it is included in Arc, is because for event based concurrency\todo{Another word instead of concurrency}, it is important to have bools, since an expression needs to be evaluated for events.

The type char is an array of characters made by using quotation marks, from the start to the end of the input. \todo{Still missing to fix, to Giovannis feedback} This is simillar to languages such as Python, C\#, JavaScript and so on. The main difference is that we have decided to call our strings, char. Char is shorthand for character, and should be recognizable for beginners and hobbyists, helping readability. The inclusion of the char type, as one on of the few types in Arc, is 
 
\subsection{Control Structures}
As mentioned, there has been taken inspiration from languages such as C, C\#, Python, JavaScript and others. There are three main types of control structure, sequence, conditional and iteration.\todo{Måske en kilde?} These are structures like if statements, for loop, while loops and so on. Since many languages use these common control structures in a similar way, we have decided to do the same. These will be described in the following sections:

\paragraph*{If statements}
The if statement is a very common conditional structure. The syntax for it, is simmilar to how many other languages structure it, with a keyword 'if', some statements and the option to add an 'else'. The reason we have not changed the structure of it, is because our language is aimed at hobbyists who might me new to coding. Therefore we have chosen a simmilar syntax to many other languages, such as C, C\# or JavaScript, so that in the future they have an easier time learning new languages.


%If statements
%\paragraph*{If statement}
%The conditional structure 'if' statement, has the terminal if, with a non-terminal expression, in parentheses, followed by non-terminal statements contained in brackets.After the if statement, there is the option to use an else statement, which follows the same structure, but without the expression. After this more statements can be made.

% The non-terminal expression, can become many different variants, and how these expressions are evaluated, will be handled in type checking. If the value is true, the conditional structure will execute the statements contained in the brackets, if not, and an else condition is made after the brackets, a jump will occur to the else condition.
% The non-terminal statements in the brackets of the if and else conditional structures can, as an expression, become many different variants. The statements, are where function calls and additional control structures are made.
% Followed by additional statements, so that more statements can be made.
% This structure is the simmilar to languages that have been used as inpiration, languages such as C, JavaScript and others, create the conditional structure in the same way.


\paragraph*{For loop}
In the same way that we have not changed much about the structure of the if statement, much has not been changed about the iteration structure, 'for' loop either. It is written with a keyword\todo{Do we call it keyword, or terminal?} 'for', then declaring a variable, another keyword 'in', with another variable which could be an array. This is the followed by some statements.
This structure is made to resemble that of Python, where a lot of the work of iterating through something is done behind the scene.

%For loop 'for' '(' TYPE_TYPEOPERATOR IDENTIFIER 'in' IDENTIFIER ')' '{' statements '}' statements 
% \subsubsection{For loop}
% The iteration structire 'for' loop, has the terminal for, with a non-terminal TYPE\_TYPEOPERATOR for the typing, and IDENTIFIER for the name of a variable, a terminal 'in' and a non-terminal IDENTIFIER, in paratheses. After this, there are non-terminal statements in brackets, followed by additional statements.
% The paramaterDeclaration is for declaring a variable, that has to match the type of the IDENTIFIER, that is to be iterated through.
% The iteration structure has especially taken inspiration from how Pythons for loop works. Where in Python, the IDENTIFIER could be an array, and the paramaterDeclaration is something that is declared, and made a placeholder for the current object in the array. In Python the paramaterDeclaration does not have a given type, however in our language the type will be declared and must match with the type of the IDENTIFIER. The reason for using a Python-like syntax, is because it is simple and rather intuitive, compared to something such as C, where an iterator is declared and used to go through an array. In our case, aswell as in Python, the work is being done in the background, so the user does not have to warry about it.


\paragraph*{While loop}
The iterative structure, 'while' loop is, as the if statement and for loop, simmilar to how other widely used programming languages use it. With a keyword 'while', with an expression in paratheses, that when evaluated to true will execute the body. The body is the set of statements in brackets. 

%While loop
% \subsubsection{While loop}
% The control structure, \todo{What do we call a while loop? Iteration structure?} 'while' loop, has the terminal while, with an expression in paratheses, followed by statements in bracket and additional statements after.
% The while loop is used to continually execute a block of code, until some condition is met. In this case, the condition is the expression, and as long as the expression evaluates to true, the block of code will be executed.

% Inspiration for the syntax of the while loop, has been taken from many different languages, since many of the languages the group has used, use a simmilar syntax. Inspiration from languages such as JavaScript, C and C\# can be seen, as the syntax matches, with a keyword 'while' with a condition in paratheses, followed by a block of code in brackets. Python is also very simmilar, the main difference is that python does not use paratheses or brackets, instead it uses colons and indents to seperate code.

\paragraph*{Switch case}
The switch case structure has been omitted from this language, the reasioning for this is simply that we saw the structure as something that can easlisy be done with if statements, and also that the structure would not be widely used by users of a hobbyists skill level. Not that it was planned to be cut from the beginning of designing the language, it stated off as being called 'when' and to be used as many other languages structure the swith case. But by discussing the design of the language, the when structure became less favored, and we therefore decided to simple ommit it from the language to also simplify the creation of the language itself. \todo{Should be read through to ensure that this is what we mean.}
