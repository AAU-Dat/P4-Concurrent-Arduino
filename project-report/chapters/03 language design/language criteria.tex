\section{Language criteria}\label{sec:languageeval}
There is no common consensus on objectively evaluating a language. The measurement of compilation speed, execution speed, and file size, address the efficiency of the language's implementation, not the language's design. Popularity could be measured but would vary significantly with time and contain several skews from bias. Popularity also does not necessarily mean that a language is generally good; for example, English is popular, but it is also a very ambiguous language.

As such, a set of prioritized language evaluation criteria has been determined based on the criteria presented in \cite{Sebesta2016}. An objective evaluation of the language based on these criteria is not practically possible \cite{Sebesta2016}; however, the criteria may work well when making decisions during language design, implementation, and evaluation. This section presents the most relevant concerns regarding the choice of relevant criteria.

The primary concern is with the programming language users - developers of Arduino programs - in this case, the Arduino hobbyist. The programming of an Arduino project may be secondary to hobbyists, who prioritize the hardware aspects of their project.

The secondary concern is the concurrency issue that follows from the primary concern - making an Arduino behave concurrently is not a trivial programming task. There are several available options to implement concurrency, but each with different issues, and none solve the issue in a general way~\cite{Restucia2022}.

The tertiary and last concern follows from the secondary issue - determining how to use concurrency to solve the problem in an Arduino. Concurrency problems can be subtle and complex, so understanding that it is even a  concurrency issue may not be immediately apparent. Thus a simple solution may be hard to spot.


\subsection{Criteria and characteristics}\label{subsec:priorityofcriteria}
Table~\ref{tab:langevalcrit} lists many characteristics that a programming language might want and their impact criteria. Sebesta~\cite{Sebesta2016} lists four criteria: readability, writability, reliability, and cost. These criteria are affected by several characteristics with varying influence and importance~\cite{Sebesta2016}. It is important to note that while these characteristics can not be measured, they are essential to keep in mind when designing the language.


\begin{table}[htb!]
    \centering
    \begin{tabular}{lccc}
        \toprule
                                & \multicolumn{3}{c}{CRITERIA}                                               \\
        \textbf{Characteristic} & \textit{Readability}         & \textit{Writability} & \textit{Reliability} \\
        \cmidrule(r){2-4}
        Simplicity              & \textbullet                  & \textbullet          & \textbullet          \\
        Orthogonality           & \textbullet                  & \textbullet          & \textbullet          \\
        Data types              & \textbullet                  & \textbullet          & \textbullet          \\
        Syntax design           & \textbullet                  & \textbullet          & \textbullet          \\
        Support for abstraction &                              & \textbullet          & \textbullet          \\
        Expressivity            &                              & \textbullet          & \textbullet          \\
        Type checking           &                              &                      & \textbullet          \\
        Exception Handling      &                              &                      & \textbullet          \\
        Restricted Aliasing     &                              &                      & \textbullet          \\
        \bottomrule
    \end{tabular}
    \caption{The three main criteria and the related characteristics~\cite{Sebesta2016}.}
    \label{tab:langevalcrit}
\end{table}


\subsubsection{Readability}
Readability describes how easy programs can be read and understood\cite{Sebesta2016}. The importance of readability is evident in the maintenance of programs, where programs with low readability are complicated and overwhelming to read.

\subsubsection{Writability}
Writability describes how easy a program is to write~\cite{Sebesta2016}. A language with good writability allows writers to express their intent more easily.

\subsubsection{Reliability}
Reliability describes how reliable a program is~\cite{Sebesta2016}. Reliability is essential, as a highly reliable program will perform correctly under all conditions.

\subsubsection{Cost}
The cost is described as a combination of several factors, such as teaching new programmers to use the language and the cost of writing the language. These costs all add up, and many things reduce this cost, such as better readability and writability, faster compile times, better reliability, and more~\cite{Sebesta2016}.

\subsection{Priority of characteristics by importance}
With these considerations in mind, we have prioritized and selected the characteristics for each criterion as we expect them to matter in this context.

The primary concern deals with considerations related to the cost criterion. Specifically, the cost associated with learning and understanding the programming language is essential. This consideration suggests that general simplicity, in the form of few but expressive language constructs and precise, consistent combination and application of the constructs is \textbf{very important}. It may also be problematic because concurrency is a complex topic.

Syntax design is also a \textbf{very important} characteristic, as seen from the secondary and tertiary concerns. The language should concisely express new constructs that are not available in the Arduino language. An aim of the syntax could be to use well known notation and keywords.

Data types are probably also an \textbf{important} characteristic. As long as the expressivity of the language is not significantly affected, the amount of data types is reduced by generalizing them. An example of this would be data types such as integers, doubles, and floats, all generalized to a single data type. Compared to the Arduino language, this reduction of data types is likely to affect overall simplicity and writability positively, and therefore cost, which is essential to hobbyists. This characteristic is also related to orthogonality - having fewer constructs and a consistent rule set for combining them is often better than having many primitives. Orthogonality is therefore also an \textbf{important} characteristic.


Expressivity is only \textbf{somewhat important} since concurrency language constructs are an aim of the language. Having too few data types is likely to harm the simplicity of the language since some things would take more work to express. On the other hand, good support for abstraction through user-defined types may enable advanced users to have greater freedom. However, readability is more important to learning than writability, and support for abstraction is \textbf{not important}.

Type checking at compile-time, especially concurrency-related issues, such as mutability, would potentially be compelling, but it depends on the rest of the design. For now, it is \textbf{less important}.

Arduino does not use the exceptions and exception handling available in the C++ language by default. The standard solution for Arduino code writers is to write code that handles the possible exceptions that may occur without the exception language constructs. For the sake of footprint, this will also be the preferred solution for this project, and exception handling is \textbf{not important}.

Aliasing refers to having two or more distinct names in a program that refers to the same memory location~\cite{Sebesta2016}. The simpler the language is, for example, not having pointers, the easier this is to restrict. While restricted aliasing is \textbf{very important}, it is expected to be easy to manage in a simple language. When dealing with concurrency, it is also essential to avoid accidental data corruption.


\begin{table}[htb]
    \centering
    \begin{tabular}{l>{\centering}p{2cm}>{\centering}p{2cm}>{\centering}p{2cm}>{\centering\arraybackslash}p{2cm}}
        \toprule
        \textbf{Characteristics}    &
        \textbf{Very important}     &
        \textbf{Important}          &
        \textbf{Somewhat important} &
        \textbf{Not important}                      \\ \midrule
        Simplicity                  & X &   &   &   \\
        Orthogonality               &   & X &   &   \\
        Data types                  &   & X &   &   \\
        Syntax design               & X &   &   &   \\
        Support for abstraction     &   &   &   & X \\
        Expressivity                &   &   & X &   \\
        Type checking               &   &   & X &   \\
        Exception handling          &   &   &   & X \\
        Restricted aliasing         & X &   &   &   \\
        \bottomrule
    \end{tabular}
    \caption{Summary of the characteristics and their importance.}
    \label{tab:priorityofcharacteristics}
\end{table}


It is worth noting that the Arduino programming language already has addressed several of the above concerns compared to C++. Examples of this can be seen in introducing new constants and the disabling of some language capabilities like exceptions and try-catch blocks from C++. The Arduino IDE is another point to consider in favor of cost concerns for the Arduino platform.
