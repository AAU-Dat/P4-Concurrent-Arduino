\subsection{Arc grammar}\label{sec:arcgrammar}
The following is the complete grammar of Arc. ANTLR grammar are similar to EBNF grammars, using regular expression notation for describing repetition.

\begin{listing}[htb!]
    \begin{minted}[label=Arc grammar]{antlr}
        start: declaration*;

        declaration
            : TYPE_TYPEOPERATOR IDENTIFIER '=' ('[' (expression (',' expression)*)? ']' | expression) ';'
            | TYPE_TYPEOPERATOR IDENTIFIER '(' (TYPE_TYPEOPERATOR IDENTIFIER (',' TYPE_TYPEOPERATOR IDENTIFIER)*)? ')' statement*
            | '#pin' IDENTIFIER '(' (PINDIGIT | NUMBER) ',' ('INPUT' | 'OUTPUT') ')' ';'
            | 'task' ('(' (TYPE_TYPEOPERATOR IDENTIFIER ( ',' TYPE_TYPEOPERATOR IDENTIFIER)*)? ')')? (('every' NUMBER) | ('when' '(' expression ')'))? statement*
            ;

        block: '{' statement* '}';

        statement
            : block
            | 'return' expression ';'
            | 'if' '(' expression ')' statement ('else' statement)?
            | 'for' '(' TYPE_TYPEOPERATOR IDENTIFIER 'in' IDENTIFIER ')' statement
            | 'while' '(' expression ')' statement
            | (TYPE_TYPEOPERATOR IDENTIFIER '=' ( '[' (expression (',' expression)*)? ']' | expression) ';')
            | IDENTIFIER ('[' NUMBER ']')? '=' ( '[' (expression (',' expression)*)? ']' | expression) ';'
            | (IDENTIFIER | ARDUINOFUNCTIONS) '(' (expression (',' expression)*)? ')' ';'
            ;

        expression
            : (NUMBER | IDENTIFIER | BOOL | CHAR)
            | (IDENTIFIER | ARDUINOFUNCTIONS) '(' (expression (',' expression)*)? ')'
            | IDENTIFIER '[' NUMBER ']'
            | '(' expression ')'
            | 'not' expression
            | expression (MULTI | DIVI) expression
            | expression (PLUS | MINUS) expression
            | expression RELATIONEQOPERATORS expression
            | expression RELATIONOPERATORS expression
            | expression 'and' expression
            | expression 'or' expression
            ;
    \end{minted}
    \caption{The \gls{cfg} grammar for Arc.}
    \label{lst:arccfg}
\end{listing}




\begin{listing}[htb!]
    \begin{minted}[label=Arc lexical rules]{antlr}
        NUMBER:                  '-'? DIGIT+ ('.' DIGIT+)?;
        PINDIGIT:                'A' [0-5];
        fragment DIGIT:          [0-9];

        IDENTIFIER:              ALPHA (DIGIT | ALPHA)*;
        fragment ALPHA:          [a-z] | [A-Z] | '_';
        
        BOOL:                    'true' | 'false';
        CHAR:                    '"' . '"';
        TYPE_TYPEOPERATOR:       'mut'? TYPE '[]'*;
        fragment TYPE:           'num' | 'bool' | 'char';

        MULTI:                   '*';
        DIVI:                    '/';
        PLUS:                    '+';
        MINUS:                   '-';
        RELATIONEQOPERATORS:     '==' | '!=';
        RELATIONOPERATORS:       '<' | '>' | '<=' | '>=';

        COMMENTS:                '//' .*? '\n' | '/*' .*? '*/'  -> skip;
        ARDUINOFUNCTIONS:        'DigitalWrite' | 'DigitalRead' | 'AnalogRead' | 'AnalogWrite';
        WS:                      [ \t\n\r]+    -> skip;
    \end{minted}
    \caption{The lexical rules for Arc.}
    \label{lst:lexicalrules}
\end{listing}




