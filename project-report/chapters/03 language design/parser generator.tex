\section{Parser generator}\label{sec:parsergenerator}
Many tools exist to automatically generate lexer and parser code, based on a grammar. Therefore, a \textit{JavaCC}, \textit{\gls{antlr}}, and \textit{CUP} discussion precedes the formal language description. The tools mentioned are only a few of those available and chosen because the group has experience with them through coursework. It is also possible to write the lexer and parser \textit{by hand}, which many do. Writing the lexer and parser by hand would give us greater control over the implementation of the language and would not require us to learn a new tool.

Each tool is evaluated using a reduced fragment of the Bims grammar from Table~\ref{tab:bimsgrammar}~\cite{Huttel2010}. For each tool, the grammar was adapted, and the tool was run with the rewritten grammar as input. Rewriting, inputting, and running the tool were compared, and a tool was selected.



\begin{table}[htb!]
  \centering
  \begin{tabular}{l}
    $n \in \textbf{Num} - \text{Numerals}$                \\
    $x \in \textbf{Var} - \text{Variables}$               \\
    $a \in \textbf{Aexp} - \text{Arithmetic expressions}$ \\
    $b \in \textbf{Bexp} - \text{Boolean expressions}$    \\
    $S \in \textbf{Stm} - \text{Statements}$              \\
  \end{tabular}
  \begin{align*}
    S & \rightarrow x \coloneqq a \mid \text{skip} \mid S_1;S_2                          \\
    b & \rightarrow a_1 = a_2 \mid a_1 < a_2 \mid \neg b_1 \mid b_1 \land b_2 \mid (b_1) \\
    a & \rightarrow n \mid a_1 + a_2 \mid a_1 * a_2 \mid a_1 - a_2 \mid (a_1)
  \end{align*}
  \caption{Sample Bims syntax~\cite{Huttel2010}}
  \label{tab:bimsgrammar}
\end{table}


\subsection{JavaCC}
JavaCC stands for Java Compiler Compiler and is an open-source parser generator and lexical analyzer developed by Oracle and written in the Java language. It is well documented and one of Java's most popular parser generators.

JavaCC takes as input a \gls{cfg} in \gls{ebnf} to generate a scanner and a parser based on the input. The parser generated is a Left-to-right, Leftmost derivation with one token lookahead (LL(1)) parser by default. However, the parser can be extended to \textit{k} lookahead for parts of the grammar if necessary~\cite{JavaCC2021}.

Setting up JavaCC was easy and only required a Java installation. Having set up, we were able to begin writing our grammar. After setting up, it became clear that writing grammar in JavaCC is simple but a bit hard to read. In JavaCC, the grammar has to be written very similar to code, whereas the syntax of other tools resembles the way it is seen in books, as shown in Table~\ref{tab:bimsgrammar}. An example of JavaCC can be seen in Listing~\ref{List:javaCC}.

\begin{listing}[htb!]
  \centering
  \begin{minted}{java}
    PARSER_BEGIN(Example)
    public class Example {
      public static void main(String args[]) throws ParseException {
        Example parser = new Example(System.in);
        parser.Input();
      }
    }
    PARSER_END(Example)
    void Input() : {}
    {
      MatchedBraces() ("\n"|"\r")* <EOF>
    }
    void MatchedBraces() : {}
    {
      "{" [ MatchedBraces() ] "}"
    }
\end{minted}
  \caption{An example of the JavaCC syntax}
  \label{List:javaCC}
\end{listing}


\subsubsection{Rewriting Bims}
Because the generated parser is recursive descent LL(1), the input grammar must not be left-recursive. However, BIMS is left-recursive and has to be rewritten; the resulting grammar is in Table~\ref{tab:bimsrewrite}.

\begin{table}[htb!]
  \centering
  \begin{tabular}{lll}
    \textit{S}  & $\to$ & $x$ $\coloneqq$ $a$ $S'$
    $\mid$  skip $S'$
    $\mid$  if $b$ then $S$ else $S$ $S'$
    $\mid$  while $b$ do $S$ $S'$                             \\
    \textit{b}  & $\to$ & $a?$ $a$ $b'$
    $\mid$  $a$ < $b'$
    $\mid$  $\neg b$ $b'$
    $\mid$  ($b$) $b'$                                        \\
    \textit{a}  & $\to$ & $n$ $a'$
    $\mid$  $x$ $a'$
    $\mid$  ($a$) $a'$                                        \\
    \textit{S'} & $\to$ & ; $S$ $S'$
    $\mid$  $\epsilon$                                        \\
    \textit{b'} & $\to$ & $\land$ $b$ $b'$ $\mid$  $\epsilon$ \\
    \textit{a'} & $\to$ & + $a$ $a'$
    $\mid$  * $a$ $a'$
    $\mid$  - $a$ $a'$
    $\mid$  $\epsilon$                                        \\
  \end{tabular}
  \caption{Bims rewritten without left recursion}
  \label{tab:bimsrewrite}
\end{table}


This \gls{cfg} has many more transition rules and takes a lot of effort to write. The \gls{cfg} of Arc would have to be written with this in mind from the start, or the rewrite might not be feasible.

\subsection{ANTLR4}
\gls{antlr} is a powerful parser generator used to process, execute, or even translate structured text or binary files. It is used worldwide by both Twitter and Oracle for parsing queries~\cite{ANTLR_About}.

One of the advantages of \gls{antlr} is the high level of support in almost any popular IDE, making it easy to work with regardless of the preferred development environment of the programmer. The ability to generate the lexer, parser, and concrete syntax tree from a single grammar file is also very appealing. The grammar is very similar to that of EBNF and utilizes a custom parsing technology called Adaptive LL(*) or ALL(*). This technology differs from the LL(*) used in ANTLR3 by analyzing the grammar dynamically at runtime rather than statically before the parser executes~\cite{Parr2014}.

\gls{antlr} is very easy to use. Install the plugin in a supported IDE of choice and follow the documentation found on the official GitHub page~\cite{ANTLR_Documentation}. Thanks to this, understanding the grammar rules and how to set it up in its respective grammar4 (.g4 file extension) is very manageable. After the grammar file has been written, it is possible to run the \gls{antlr} tool, automatically generating several files, such as a parser, a lexer, and a set of tokens. Then the generated code can be compiled with the relevant compiler. This process also provides the developer with a visual representation of the created parse tree or a tree in a LISP-like text form and a list of tokens found - all of these options are accessible through optional flags during compilation. The fourth major version of \gls{antlr} introduces a lot of new capabilities compared to ANTLR3, which helps reduce the learning curve by making the development of grammars and language applications easier - in part thanks to the powerful extensions and tools included in their plugins.


\begin{listing}[htb!]
  \begin{minted}{antlr}
    grammar Bims;
    S:    VAR ':=' NUM | S ';' S;
    B:    A '=' A;
    A:    NUM | A ('+' | '*' | '-') A | '(' A ')';
    
    NUM:  [0-9]*;
    VAR:  [a-z]*;
  \end{minted}
  \caption{Bims grammar from Table~\ref{tab:bimsgrammar} with ANTLR syntax}
  \label{tab:antlrexample}
\end{listing}


\subsection{CUP}
CUP stands for Construction of Useful Parsers and is a Look-Ahead Left-to-right, Rightmost derivation (LALR) parser generator for Java~\cite{cupParserGenerator}. CUP implements a standard LALR(1) parser generation. Documentation for CUP recommends using the scanner generator JFlex, as a Lexical Analyzer Generator. Listing~\ref{list:cup} is an example of how CUP was used.


\begin{listing}[htb!]
  \centering
  \begin{minted}{java}
    /* Simple +/-*/ expression language; parser evaluates constant expressions
    import java_cup.runtime.*;

    parser code {:
      parser p = new parser(new Scanner(new FileReader(fileName)));
      Object result = p.parse().value;  
      :}

    /*define how to connect to the scanner!*/
    init with {: p.init(); :};
    scan with {: return p.next_token(); :};

    terminal PLUS;
    terminal Integer NUMBER;

    non terminal exp;
    exp ::= exp PLUS NUMBER | NUMBERM;
  \end{minted}
  \caption{An example of the CUP syntax}
  \label{list:cup}
\end{listing}


In Listing~\ref{list:cup} we created a simple grammar that could add two integers together; to try using CUP, and to understand how it works.


\subsubsection{JFlex}
JFlex is an abbreviation for Java Flex. Flex is an abbreviation for Fast Lexical Analyzer Generator. As the name suggests, it is a tool for generating fast lexers. It makes adjustments to the size and speed of the generated lexer possible.

Listing~\ref{list:jflex} is an example of how JFlex looks, in conjunction with our previously mentioned CUP file.


\begin{listing}[htb!]
  \centering
  \begin{minted}{text}
    LineTerminator     = \r|\n|\r\n
    WhiteSpace         = {LineTerminator} | [\t\f]
    DecIntergerLiteral = 0 | [1-9][0-9]*

    /* literals */
    {DecIntergerLiteral}    {return symbol(sym.NUMBER);}
    \                       {string.setLenght(0); yybegin(STRING);}
    /*whitespace*/ 
    {WhiteSpace}            {/*ignore */}  
    /*operators*/
    "+"                     {return symbol(sym.PLUS);}
  \end{minted}
  \caption{An example of the JFlex syntax}
  \label{list:jflex}
\end{listing}


\subsection{Result}
CUP will not be used in the project, as it proved difficult, combined with a lack of documentation, to connect the parser generator with the lexical analyzer generator. In contrast, other tools, such as JavaCC and \gls{antlr}, automatically create the lexical analyzer.

Based on our experiences with the different compiler compilers, we have chosen to work with \gls{antlr}. We chose \gls{antlr} primarily because of how powerful it is and the great extensions it provides and because direct left recursion is not a problem for it, as it is for JavaCC. In particular, these factors could help us develop our language and compiler efficiently. It does commit us to a LL parser which is not as expressive as an LR parser, but we do not believe this will be a problem for our language since it will be relatively compact.

\gls{antlr} allows us to iterate and test grammars and grammar structures quickly, so we use the tool rather than write the lexer and parser by hand. This iteration process is practical as implementing the lexer and parser is only a part of the implementation.