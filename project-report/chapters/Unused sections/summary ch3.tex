\subsection{Syntax summary}
In this chapter there has been discussed about the criteria for Arc, what was important when creating the language, and what was less important. From this a priority table was produced, that showcases the importance of different characteristics. Following this a discussion of what parser generator to use, was had. The parser generators were JavaCC, ANTLR, and CUP. Pros and cons of the different parser generators were brought up, to find the parser generator that was be suited for the current needs. ANTLR was chosen as the parser generator to be used for Arc, since it proved easy to use and had good documentation. \todo{Eh...} From this, a grammar could be made, the criteria for it had been made and a parser generator had been chosen, the grammar was shown. Following this, inspiration for that grammar was discussed, the thoughts and choices that had been made, and why they had been made. Then the concurrency structures of Arc were discussed, how they are made and why they are structured the way they are. From this semantics of Arc will come, to discus the meaning behind the syntax. \todo{Overvej om summary er relevant}


