\section{Arduino}\label{sec:arduino}
Arduino is an open-source platform that enables users to easily and simply create small projects. Including projects such as an LED lighting up when a button is pushed, or building a thermostat to display the current ambient temperature. It is classified as a microcontroller board, which is a small computer that enables users to read some input and turn it into an output, and between this apply some logic. Arduino can be split into two categories, the physical and software:

The physical part of the Arduino is the board. There exist many types of boards that have different components and specs, but the most common board is the Arduino UNO. The Arduino UNO SET provides a USB cable to connect to a user's PC, a power input, some digital and analog pins, the microcontroller itself, the ATmega328, and various other components \cite{WhatArduinoLearn}.

The software that Arduino makes use of is its IDE, which enables users to write code, also called sketches, which can be uploaded on the board and then be run. The code for Arduino is, in some ways, a simplified version of C++, which has integrated methods specifically made for Arduino.

The board is made to be inexpensive and simple to use, so it is possible for people with little to no experience working with electronics to pick up and start making projects. As in the past, it was necessary to have in-depth knowledge about microcontrollers and the like, to be able to create such projects. The Arduino IDE is also cross-platform, enabling users of Linux, Windows, and Mac OS to run the Arduino software \cite{WhatArduino}.

\subsection{Arduino Language}\label{sec:arduinolanguage}
The Arduino language is a smaller version of C++. It holds most of the known semantics made from C++, but with some differences, which will be covered in this section \cite{WhatArduino}. A program in Arduino is called a sketch, Arduino sketch normally follows a basic structure, that consists of implementing the procedures "setup()" and "loop". "setup()" is called once at the beginning of the execution of the program; it is meant to initialize the data structure and pins \cite{ArduinoSetup}. After the setup procedure, the loop() function beings. The loop() function is a loop that runs while the Arduino is on, and implements the Arduino behavior \cite{ArduinoLoop}.

\subsection{Who uses it}
Arduino is used by a wide range of people, such as students, hobbyists, programmers, and professionals \cite{WhatArduino}.
Many different groups of people use the Arduino for different purposes. Some are focused on the learning aspects of the Arduino, others are more interested in easily designing concepts for a product. Students can use the board to learn the basics of electronics, hobbyists use it to build projects, and professionals use it to design product concepts \cite{ArduinoUsedReal2018}.

For this report, the focus is on hobbyists who use the Arduino to create projects and want to learn the basics of electronics. The main concern regarding this group of users is that they spend a limited amount of time working with Arduino, as the time they spend with it will be in their leisure time. Moreover, hobbyists might have limited coding proficiency or be complete beginners. \todo{A little more detail on what we define as our target audience - hobbyists.}
We characterize the users we want to focus on as hobbyists who want to do as much as they can with their limited time and limited coding proficiency.


%https://roboticsbackend.com/is-arduino-used-in-real-life-products/
%https://all3dp.com/2/best-arduino-uses/

%Good for students, hobby folk and for building a concept for a product, but price for a board is too high for a product.
%Why do we want to focus on hobbyists

\subsection{What do hobbyists use the Arduino for}\todo{Write into definition of our target audience}

Hobbyists use Arduino for many different things, a Google search will show a vast amount of project ideas to people showing off their projects and maybe guiding people through to make their version.  Such as this list of 10 uses for the Arduino \cite{Top10Arduino2021}.
The list includes projects that range from a simple thermostat to display the current temperature, to an automatic gardening device to water plants.

This shows just some of the projects that hobbyists could be interested in making, but for some projects, it might be useful for hobbyists to use concurrency. 


%Få nævnt brug af concurrency

%https://all3dp.com/2/best-arduino-uses/