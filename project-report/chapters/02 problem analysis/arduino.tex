\section{Arduino}\label{sec:arduino}
Arduino is an open-source electronics platform that enables users to quickly create small microcontroller projects through easy-to-use hardware and software \cite{WhatArduino}. Multiple variants of Arduino boards exist, with different components and specifications. The UNO, Due, Mega2560, Micro, Leonardo, Zero, Mini, and UNO WiFi are within the classic family of Arduino boards. Because of Arduino being an open-source project, other companies are free to use the specifications to provide third-party implementations of the boards.

An Arduino UNO is used as the reference board for this project because \gls{aau} has provided one. All physical tests and examples will run on this board, and the implementation will depend on the Arduino UNO specification henceforth.

The software for the Arduino is the official Arduino \gls{ide}, in which developers can write code in the Arduino programming language. The \gls{ide} and the programming language is built on Wiring, which builds on Processing \cite{WhatArduino,WiringOrg}.

\subsection{The Arduino programming language}\label{subsec:arduinoprogramminglanguage}
The Arduino programming language closely resembles the C programming language and is, in fact, a set of C/C++ functions \cite{ArduinoSupportC}. It contains most of the constructs of C++, with some additional functions added and a few features, turned off in the compiler, such as exceptions \cite{Nongnuorg}.

\subsubsection{Sketches}
A program written in the Arduino gls{ide} and programming language is called a sketch. A sketch follows a basic structure that consists of implementing the procedures "setup()" and "loop()". Setup is a procedure called once during a run - when the Arduino is turned on or reset; it initializes variables, pin modes, libraries, and some additional links. After setup, the loop begins. The loop procedure is a loop that runs for the duration of the runtime of the program and contains the logic of the project \cite{ArduinoLanguage}.

\subsection{Who uses the Arduino?}\label{subsec:whouses}
A wide range of people, such as students, hobbyists, children, programmers, and professionals, use Arduino for many different purposes \cite{WhatArduino}. Some are focused on the learning aspects of the Arduino - architecture, code, and embedded systems, while others are more interested in designing product concepts. Students and children can use the board to learn the basics of electronics; hobbyists may use it to build personal \gls{diy} projects, while professionals often use it to design product concepts \cite{WhatArduino}.

Designing a programming language for dedicated programmers or professionals may require a profound understanding of many underlying details of the Arduino platform itself. Providing a good language for these groups can potentially detract from the project's purpose - which is to design a programming language - because it may take more time than is available to obtain this knowledge. Thus, this is not the target group. The project also will not deal with a programming language designed for children, as this would require an understanding of pedagogical tools, which is outside the scope of a computer science education.

The hobbyist is, however, an excellent group for which to design our implementation for. Because hobbyists spend their leisure time on Arduino projects, the primary concerns are their limited available time and the potential for frustration during a project. Moreover, hobbyists might have limited programming proficiency or be complete beginners. The standard Arduino language already alleviates many problems, but not in regards to concurrency. The user group for the remainder of this report is hobbyists who want to do as much as they can with their limited time and limited coding proficiency - specifically related to concurrency.