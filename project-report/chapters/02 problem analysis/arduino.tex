\section{Arduino}\label{sec:arduino}
Arduino is an open-source electronics platform that enables users to easily create small microcontroller projects through easy-to-use hardware and software \cite{WhatArduino}. Multiple variants of Arduino boards exist, with different components and specifications. Within the classic family of Arduino boards are the UNO, Due, Mega2560, Micro, Leonardo, Zero, Mini and UNO WiFi. Because the Arduino is an open-source project, other companies are free to use the specifications to provide third party implementations of the boards.

Because \gls{aau} has provided an Arduino UNO, an Arduino UNO is used as the reference board for this project. All physical tests and examples are run on this board, and the implementation will depend on the Arduino UNO specification henceforth.

The software for the Arduino is the official Arduino IDE, where code in the Arduino programming language can be written. Both the IDE and the programming language is based on Wiring, which builds on Processing \cite{WhatArduino,WiringOrg}.

\subsection{The Arduino programming language}\label{subsec:arduinoprogramminglanguage}
The Arduino programming language closely resembles the C programming language, and is in fact a set of C/C++ functions \cite{ArduinoSupportC}. It contains most of the constructs of C++, with a some additional functions added, and a few features, such as exceptions, turned off in the compiler \cite{Nongnuorg}.

\subsubsection{Sketches}
A program written in the Arduino IDE and programming language is called a sketch. A sketch follows a basic structure, that consists of implementing the procedures "setup()" and "loop". Setup is a procedure called once during a run - when the Arduino is turned on or reset; it is used to initialize variables, pin modes, libraries, and so on. After setup, the loop begins. The loop procedure is a loop that runs for the duration of the runtime of the program, and contains the logic of the project \cite{ArduinoLanguage}.

\subsection{Who uses the Arduino?}\label{subsec:whouses}
Arduino is used by a wide range of people, such as students, hobbyists, children, programmers and professionals, and with many different purposes \cite{WhatArduino}. Some are focused on the learning aspects of the Arduino - architecture, code, etc., while others are more interested in designing product concepts. Students and children can use the board to learn the basics of electronics, hobbyists may use it to build personal \gls{diy} projects, while professionals often use it to design product concepts \cite{WhatArduino}.

Designing a programming language for dedicated programmers or professionals may require a very deep understanding of a lot of underlying details of the Arduino platform itself, to provide a proper language for these groups. This can potentially detract from the purpose of the project - which is to design a programming language - because it may take more time than is available to obtain this knowledge, and thus this is not the target group. The project also won't deal with a programming language designed for children, as this would require understanding of pedagogical tools, which is outside the scope of the computer science education.

The hobbyist is however a promising group to design for. Because hobbyists spend their leisure time on Arduino projects, the primary concerns are in regards to their limited available time, and the potential for frustration during a project. Moreover, hobbyists might have limited programming proficiency, or be complete beginners. The standard Arduino language already alleviates a lot of problems, but not in regards to concurrency. As such, the user group for the language design in this report is characterized as: hobbyists who want to do as much as they can with their limited time and limited coding proficiency - specifically in relation to concurrency.
