\section{Concurrency}\label{sec:concurrency}
Note: Operating systems, libraries, scheduling, and differences between the options.


\blockcquote{Bryant2016}{We use the term concurrency to refer to the general concept of a system with
multiple, simultaneous activities.}

The scheduler is usually what takes care of these choices and prioritization

Concurrency is used for things like coordinating the execution of processes to maximize throughput. Concurrency is not to be confused with parallelism, which requires multiple cores because you have to do several things at the same time. Concurrency, on the other hand, does not need to be at the same time, just in the right order at the correct times, so it is possible to do with a single core. Since the Arduino has only a single core, this project deals with concurrency without parallelism.

It is possible to achieve concurrency on Arduino, but not without scheduling of some kind. Scheduling is to schedule different tasks to run on the same CPU, fooling users to think that they all run at the same time, while in fact it is running different tasks very fast and switching between them. This scheduling is not built into Arduino so the project will have to decide on how to achieve it. Part of the problem is choosing how to do this, while also keeping the end-user in mind. A hobbyist typically tries to achieve concurrency through the standard functions in Arduino, but many of these functions are not ideal for concurrency.

Several tutorials\todo{Maybe include references or examples?} on how to implement different techniques to achieve some sort of concurrency in Arduino exist. These techniques include Milis() and Interrupt(). Milis() uses the milis function to tell the time, while not pausing to achieve the ability to wait and not do unnecessary tasks at all times. The interrupt() function, on the other hand, cancels the process if something happens, and this can be used to change between different tasks.


\subsection{Concurrency for hobbyist} \feedback{Try to polish your argument}
Concurrency is not achieved effortlessly with Arduino, as an additional layer of programming complexity is necessary. This is, of course, a problem for the hobbyist who is not the most experienced programmer, since it makes the problem of getting the Arduino to do what they want even harder. It also increases the time the hobbyist spends implementing a solution that seems relatively intuitive. To help the hobbyist achieve correct and functional concurrency easily and simply, this is what the project will try to do.\todo{Early conclusion and problem statement. Perhaps too early? If not, it is a bit superficial}.