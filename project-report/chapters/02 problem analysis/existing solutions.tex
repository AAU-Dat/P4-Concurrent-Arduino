\section{Existing solutions}\label{sec:existingsolutions}
In this section, some ways of achieving concurrency on an Arduino is explored. Emphasis is put on the immediate advantages and disadvantages related to each model, when considering the impact on the hobbyist users, by which is meant the new challenges, if any, introduced by the model in question.

A small sample project is implemented in each model for the sake of comparison: The builtin \gls{led} of the Arduino is set to blink - turn on and off - every half second. Concurrently, the state of a button is read and printed continously. The example will be implemented in the Arc programming language as well. The schematic for the project is seen in Figure~\ref{fig:exampleprojectschematic}.


\begin{figure}[htb!]
  \centering
  \missingfigure[figwidth=0.8\textwidth]{Insert image of Arduino schematic?}
  %\includegraphics[width=0.8\textwidth]{example-image-a}
  \caption{Example project schematic}
  \label{fig:exampleprojectschematic}
\end{figure}


The first discussion is about the prospect of using programming libraries to achieve concurrency on an Arduino. The second discussion is about the possibility of installing an \gls{os} on the Arduino, and building on the \gls{os}-exposed concurrency abstractions.

\subsection{Concurrency through Arduino libraries}\label{subsec:arduinolibraries}
In this section, the \textit{Protothreads} and \textit{Eventually} libraries are explored and compared. Of particular concern is the quality of the documentation, the model of concurrency, and the overhead of usage on the Arduino.

\subsubsection{Protothreads}
Protothreads is a library for the C language, which has been implemented as a library for Arduino~\cite{Artin2020}. It is based on Adam Dunkels protothreads~\cite{AdamDunkelProtothreads}.


\begin{listing}[htb!]
  \centering
  \begin{minted}[label=Protothreads example]{arduino}
    #include "protothreads.h"
    #define BUTTON_PIN 12
    int buttonState = 0;  // variable for reading the pushbutton status
    pt ptBlink, ptButton;

    int blinkThread(struct pt *pt) {
      PT_BEGIN(pt);
      for (;;) {
        digitalWrite(LED_BUILTIN, HIGH); // turn the LED on (HIGH is the voltage level)
        PT_SLEEP(pt, 1000);
        digitalWrite(LED_BUILTIN, LOW); // turn the LED off by making the voltage LOW
        PT_SLEEP(pt, 1000);
      }
      PT_END(pt);
    }

    int buttonThread(struct pt *pt) {
      PT_BEGIN(pt);
      for (;;) {
        buttonState = digitalRead(BUTTON_PIN);
        Serial.println(buttonState);
        PT_YIELD(pt);
      }
      PT_END(pt);
    }

    void setup() {
      PT_INIT(&ptBlink);
      PT_INIT(&ptButton);
      Serial.begin(9600);
      pinMode(LED_BUILTIN, OUTPUT);
      pinMode(BUTTON_PIN, INPUT);
    }

    void loop() {
      PT_SCHEDULE(blinkThread(&ptBlink));
      PT_SCHEDULE(buttonThread(&ptButton));
    }
  \end{minted}
  \caption{A small example on how a Protothreads can be implemented}
  \label{lst:protothreadsexample}
\end{listing}


\blockcquote{Artin2020, AdamDunkelProtothreads}{Protothreads provides a blocking context on top of an event-driven system, without the overhead of per-thread stacks. The purpose of protothreads is to implement sequential flow of control without complex state machines or full multi-threading.}

The pitch for Protothreads is promising, and directly addresses the concern about overhead. However it does not describe the model of concurrency in great detail. Fortunately the documentation is excellent on both the Arduino implementation~\cite{Artin2020} as well as the general Protothreads specification~\cite{AdamDunkelProtothreads}.


\begin{table}[htb!]
  \centering
  \begin{tabular}{lll>{\bfseries}l}
    \toprule
    Features             & Events & Threads & Protothreads \\ \midrule
    Control structures   & no     & yes     & yes          \\
    Debug stack retained & no     & yes     & yes          \\
    Implicit locking     & yes    & no      & yes          \\
    Preemption           & no     & yes     & no           \\
    Automatic variables  & no     & yes     & no
  \end{tabular}
  \caption{Qualitative comparison between events, threads and protothreads ~\cite{dunkels05using}}
  \label{tab:protothreadscomparison}
\end{table}


Protothreads uses a cooperative form of concurrency, which means it is us to the user to synchronize the program. This means that a program written with protothreads is partially event-driven and blocking - it must finish, or pause, explicitly, before moving on to the next task. An overview comparison between protothreads and event based and thread based concurrency can be seen in Table~\ref{tab:protothreadscomparison}, which shows that protothreads are a mix of the traditional event based and thread based models.

Protothreads is also implemented on a single stack with stack rewinding, unlike traditional multithreadings which has a stack per thread. This is the reason for the low overhead of Protothreads. On the Arduino, this is achieved through utilization of \textbf{local continuations} - threads are simply a struct with an unsigned short - together with macros that expand to a switch statement with a number of returns. The short contained in the thread is the set and compared against the switch defined by the macros to set the state and continuation.

Because threads are a struct with a short, they have a size of only two bytes. This means that there is no hidden memory cost during the execution of the program. However, the implementation details of the Protothread library does have a few effects on how to write programs when using it.

First, a protothread only saves the short across blocking calls. This means that local variables inside a thread are not preserved - a rule of thumb from the designer is that local variables simply should not be used inside a protothread. Instead, it is prudent to use global variables, if data should be preserved.

Secondly, the scheduling of protothreads uses a switch statement in a way that restricts the rest of the code, such that code cannot use switch statements with protothreads.

Lastly, when using protothreads, programmers some of the Arduino functions, like delay(), should not be used. Protothreads are already blocking if written to be, so blocking the cpu with delay() would prevent protothreads from executing correctly~\cite{AdamDunkelProtothreads}.

The implementation of our sample project with protothreads can be seen in Listing~\ref{lst:protothreadsexample}.


%---Links to Eventually c++ Libary---
%1. https://github.com/johnnyb/Eventually
%2. https://www.arduino.cc/reference/en/libraries/eventually/
% What is Eventually C++ Library?
% Advantages and disadvantages of Eventually Library (C++ issues)

\subsubsection{Eventually}
Eventually is another library for Arduino, implementing an event-driven concurrency model in C++. It is written by Jonathan Bartlett and Ben Jenkinson.


\begin{listing}
  \begin{minted}[label=Eventually example]{arduino}
    #include <Eventually.h>
    #define BUTTON_PIN 12
    bool pinState = true;
    EvtManager mgr;

    bool blinker() {
      mgr.resetContext();
      mgr.addListener(new EvtTimeListener(1000, true, (EvtAction)blink_pin)); 
      mgr.addListener(new EvtPinListener(BUTTON_PIN, (EvtAction)digital_read));
    }

    void blink_pin() {
      if (pinState == true) {
        digitalWrite(LED_BUILTIN, HIGH);
      } else {
        digitalWrite(LED_BUILTIN, LOW);
      }
      pinState = !pinState;
    }

    void digital_read() {
        int sensorVal = digitalRead(BUTTON_PIN);
        Serial.println(sensorVal);
        delay(1);
    }

    void setup() {
      Serial.begin(9600);
      pinMode(LED_BUILTIN, OUTPUT);
      pinMode(BUTTON_PIN, INPUT);
      blinker();
    }

    USE_EVENTUALLY_LOOP(mgr)
  \end{minted}
  \caption{A small program on how Eventually can be implemented}
  \label{lst:eventuallyexample}
\end{listing}


\blockcquote{bartlettEventually2022Bartlett}{The goal of Eventually is to make a more event-oriented environment for Arduino programming. To make it actually easy to use to build worthwhile projects.}

Based on the pitch, the concurrency model implemented in the Eventually library is somewhat similar to the event loop model employed by the Node runtime for JavaScript~\cite{NodeJSdocs}. It is simpler, less powerful, and smaller - but similar.

The documentation for Eventually is very sparse, compared to Protothreads, which makes it difficult to obtain information outside reading the sourcecode. In short, it is a callback oriented model with two inbuilt listener types - \textit{timed listener} and \textit{pin listener} - with the option to extend the model by writing additional listeners. These listeners listen for timing events or pinstate events, and fire the relevant callback function when the event is registered.

One of the issues with a callback based approach is the notion of callback hell, callbacks inside callbacks inside callbacks, which can become complicated and difficult to read. In Node this is addressed with Promises, but no such construct is available in the Eventually library.

The implementation of our sample project with eventually can be seen in Listing~\ref{lst:eventuallyexample}. It shows the use of 



To give a better understanding of how the Eventually C++ library is working, in this section there has been taken an example from a GitHub page ~\cite{bartlettEventually2022Bartlett} where there can be seen, a code example on how to use the Eventually library. Since it is a C++ library it would work on any Arduino board, which includes the one the group has acquired ~\cite{bartlettEventually2022Bartlett}





\subsection{Achieving Concurrency with an Operating System}\label{subsec:arduinoos}



%---Links to Free rtos---
%1. https://github.com/feilipu/Arduino_FreeRTOS_Library
%   Forklar hvad link går ud på
\subsubsection{FreeRTOS}
Free Real-Time Operating System (abbreviated to FreeRTOS) is an operating system specifically designed for microcontrollers and microcomputers, such as the Arduino. It has been developed in partnership with the leading chip companies in the world, over more than 18 years, and with a special emphasis on reliability, accessibility and ease of use ~\cite{AboutRTOS}. This leads itself well to our project, while we are targeting hobbyists. FreeRTOS utilises preemptive scheduling ~\cite{SchedulingRTOS}, which means that it implements a scheduler to be responsible for deciding which tasks to do in which order.



\begin{listing}[htb!]
  \centering
  \begin{minted}[label=FreeRTOS example]{arduino}
    #include <Arduino_FreeRTOS.h>
    #define BUTTON_PIN 12

    void TaskBlink(void *pvParameters){ // This is a task.
      (void) pvParameters;
      pinMode(LED_BUILTIN, OUTPUT);
      for (;;) { // A Task shall never return or exit.
        digitalWrite(LED_BUILTIN, HIGH);
        vTaskDelay( 1000 / portTICK_PERIOD_MS );
        digitalWrite(LED_BUILTIN, LOW);
        vTaskDelay( 1000 / portTICK_PERIOD_MS );
      }
    }

    void TaskAnalogRead(void *pvParameters){ // This is a task.
      (void) pvParameters;
      for (;;) {
        int sensorValue = digitalRead(BUTTON_PIN);
        Serial.println(sensorValue);
        vTaskDelay(1);
      }
    }

    void setup() {
      Serial.begin(9600);
      while (!Serial) {;}
      xTaskCreate(TaskBlink, "Blink", 128, NULL, 2, NULL);
      xTaskCreate(TaskAnalogRead, "AnalogRead", 128, NULL, 1, NULL);
    }

    void loop(){}
\end{minted}
  \caption{A small example of a possible implementation of Free RTOS.}
  \label{lst:freeftosexample}
\end{listing}

%and Preemptive concurrency forms \cite{UsingFreeRTOSMultitasking}, the preemptive concurrency form is priority-based, like the cooperative concurrency form the preemptive concurrency form has to prioritise the tasks in the program, but a task has to be completed in the time period which the scheduler has given task, it then switches to another task and if the first task was not completed it switches back \cite{Windows}.
%the library has a scheduler to set up when the tasks have to be executed, the programmer can also set up when the Arduino has to do a specific task, if no specific timed workflow, the scheduler assigns the priority of the tasks \cite{UsingFreeRTOSMultitasking}. 

%---Links to Simba library---
% 1. https://simba-os.readthedocs.io/en/latest/
%   Forklar hvad link går ud på
% 2. https://all3dp.com/2/best-arduino-operating-system/
%   Forklar hvad link går ud på
%\subsubsection{Simba}



%---Links to TaskManagerIO library---
% 1. https://github.com/davetcc/TaskManagerIO
%   Forklar hvad link går ud på
% 2. https://all3dp.com/2/best-arduino-operating-system/
%   Forklar hvad link går ud på
%\subsubsection{TaskManagerIO }




It is also possible to install an \gls{os} on the Arduino, and obtain a scheduler (and other things) that way.

\subsection*{Deprecated at the moment}
\subsection{Models of concurrency}\label{subsec:modelsofcon}
% might be relevant to explain the concurrency models of an OS, to compare the choices.
This interleaving is commonly handled through an \gls{os}, which manages the hardware resources ~\cite{Bryant2016}. A few different approaches for interleaving processes are supported by most \glspl{os}:

\subsubsection{Processes}
In this model the kernel, the portion of the \gls{os} code that resides in memory while the system is running, schedules and maintains each logical control flow, called a process. Each process has its own virtual address space, and therefore requires a mechanism for interprocess communication ~\cite{Bryant2016}.

\subsubsection{I/O multiplexing}
When an application explicitly schedules its own logical control flow, in the context of a single kernel process, you have I/O multiplexing. In the application, each logical control flow is modelled as a state machine, with the transitions defined and managed by the application code. Since this model is a single process, control flows share virtual address space ~\cite{Bryant2016}.

\subsubsection{Threads}


\subsection{Summary}

