\section{Existing solutions}\label{sec:existingsolutions}
This section explores some ways of achieving concurrency on an Arduino. Emphasis is put on the immediate advantages and disadvantages, related to each model when considering the impact on the hobbyist users, introduced by the model in question.

A small sample project is implemented in each model for the sake of comparison: The builtin \gls{led} of the Arduino is set to blink - turn on and off - every half second. Concurrently, the state of a button is read and printed continously. The example will be implemented in the Arc programming language as well. Figure~\ref{fig:exampleprojectschematic} shows the schematic for the project.


\begin{figure}[htb!]
  \centering
  \includegraphics[width=0.5\textwidth]{figures/Example_Project}
  \caption{Example project schematic.}
  \label{fig:exampleprojectschematic}
\end{figure}


The first discussion is about the prospect of using programming libraries to achieve concurrency on an Arduino. The second discussion is about the possibility of installing an \gls{os} on the Arduino and building on the \gls{os}-exposed concurrency abstractions.

\subsection{Concurrency through Arduino libraries}\label{subsec:arduinolibraries}
In this section, the \textit{Protothreads} and \textit{Eventually} libraries are explored and compared. Of particular concern is the quality of the documentation, the model of concurrency, and the overhead of usage on the Arduino.

\subsubsection{Protothreads}
Protothreads is a library for the C language, which has been implemented as a library for Arduino~\cite{Artin2020}. It is based on Adam Dunkels Protothreads~\cite{AdamDunkelProtothreads}.


\begin{listing}[htb!]
  \centering
  \begin{minted}[label=Protothreads example]{arduino}
    #include "protothreads.h"
    #define BUTTON_PIN 12
    int buttonState = 0;  // variable for reading the pushbutton status
    pt ptBlink, ptButton;
    int blinkThread(struct pt *pt) {
      PT_BEGIN(pt);
      for (;;) {
        digitalWrite(LED_BUILTIN, HIGH); // turn the LED on (HIGH is the voltage level)
        PT_SLEEP(pt, 1000);
        digitalWrite(LED_BUILTIN, LOW); // turn the LED off by making the voltage LOW
        PT_SLEEP(pt, 1000);
      }
      PT_END(pt);
    }
    int buttonThread(struct pt *pt) {
      PT_BEGIN(pt);
      for (;;) {
        buttonState = digitalRead(BUTTON_PIN);
        Serial.println(buttonState);
        PT_YIELD(pt);
      }
      PT_END(pt);
    }
    void setup() {
      Serial.begin(9600);
      PT_INIT(&ptBlink);
      PT_INIT(&ptButton);

      pinMode(buttonPin, INPUT);
    }
    void loop() {
      PT_SCHEDULE(blinkThread(&ptBlink));
      PT_SCHEDULE(buttonThread(&ptButton));
    }
  \end{minted}
  \caption{Protothreads implementation of the sample project.}
  \label{lst:protothreadsexample}
\end{listing}


\blockcquote{Artin2020, AdamDunkelProtothreads}{Protothreads provides a blocking context on top of an event-driven system, without the overhead of per-thread stacks. The purpose of protothreads is to implement sequential flow of control without complex state machines or full multi-threading.}

The pitch for Protothreads is promising and directly addresses the concern about overhead. However, it does not describe the model of concurrency in great detail. Fortunately, the documentation is excellent on both the Arduino-specific implementation~\cite{Artin2020} and the general Protothreads specification~\cite{AdamDunkelProtothreads}.


\begin{table}[htb!]
  \centering
  \begin{tabular}{lll>{\bfseries}l}
    \toprule
    Features             & Events & Threads & Protothreads \\ \midrule
    Control structures   & no     & yes     & yes          \\
    Debug stack retained & no     & yes     & yes          \\
    Implicit locking     & yes    & no      & yes          \\
    Preemption           & no     & yes     & no           \\
    Automatic variables  & no     & yes     & no
  \end{tabular}
  \caption{Qualitative comparison between events, threads, and Protothreads~\cite{dunkels05using}.}
  \label{tab:protothreadscomparison}
\end{table}


Protothreads uses a cooperative form of concurrency, which means it is up to the user to synchronize the program. This model means that a program written with Protothreads is partially event-driven and blocking - it must finish, or pause, explicitly before moving on to the next task. An overview comparison between Protothreads, event-based and thread-based concurrency is available in Table~\ref{tab:protothreadscomparison}, which shows that Protothreads mix the traditional event-based and thread-based models.

Protothreads are also implemented on a single stack with stack rewinding, unlike traditional multithreading, which has a stack per thread. This implementation is the reason for the low overhead of Protothreads. On the Arduino, this is achieved by utilizing \textbf{local continuations} - threads are simply structs with an unsigned short - and macros that expand to a switch statement with several returns. The short in the protothread is set and compared against the expanded switch throughout the program running \cite{AdamDunkelProtothreads}.




Because Protothreads are structs with only a short, they have a size of only two bytes. This model means that there is no hidden memory cost during the program's execution. However, the implementation details of the Protothreads library has little effect on how to write programs when using it.

First, a protothread only saves the short across blocking calls, which means that local variables inside a thread are not preserved - a rule of thumb from the designer is that the user should not use local variables inside a protothread. Instead, it is prudent to use global variables if the user wants to preserve data.

Secondly, the scheduling of Protothreads uses a switch statement in a bastardized way that restricts the rest of the code, such that code cannot contain switch statements.

Lastly, some of the Arduino functions - like delay() - should not be used when using Protothreads. Protothreads are already blocking, so blocking the \gls{cpu} with delay() could potentially prevent Protothreads from executing correctly~\cite{AdamDunkelProtothreads}.

Listing~\ref{lst:protothreadsexample} shows the protothread implementation of the sample project.


\subsubsection{Eventually}\label{subsec:eventually}
Eventually~\cite{bartlettEventually2022Bartlett} is another library for Arduino, implementing an event-driven concurrency model in C++. It was created and maintained by Jonathan Bartlett and Ben Jenkinson.


\begin{listing}[htb!]
  \begin{minted}[escapeinside=\#\#, highlightlines={18-19,28-29},label=Eventually example]{arduino}
    #include <Eventually.h>
    #define BUTTON_PIN 12
    bool pinState = true;
    EvtManager mgr;
    void digital_read() {
      int sensorVal = digitalRead(BUTTON_PIN);
      Serial.println(sensorVal);
      delay(1);
    }
    void blink_pin() {
      if (pinState == true) {
        digitalWrite(LED_BUILTIN, HIGH);
      } else {
        digitalWrite(LED_BUILTIN, LOW);
      }
      pinState = !pinState;
    }
    EvtPinListener *reader1 = new EvtPinListener(BUTTON_PIN, (EvtAction)digital_read);#\label{minted:eventually1}#
    EvtPinListener *reader2 = new EvtPinListener(BUTTON_PIN, (EvtAction)digital_read);#\label{minted:eventually2}#
    bool blinker() {
      mgr.resetContext();
      mgr.addListener(new EvtTimeListener(1000, true, (EvtAction)blink_pin));
      mgr.addListener(reader1);
      mgr.addListener(reader2);
    }
    void setup() {
      Serial.begin(9600);#\label{esc:test}#
      reader1->targetValue = HIGH; #\label{minted:eventually3}#
      reader2->targetValue = LOW; #\label{minted:eventually4}#
      blinker();
    }
    USE_EVENTUALLY_LOOP(mgr)
  \end{minted}
  \caption{Eventually implementation of the sample project.}
  \label{lst:eventuallyexample}
\end{listing}


\blockcquote{bartlettEventually2022Bartlett}{The goal of Eventually is to make a more event-oriented environment for Arduino programming. To make it actually easy to use to build worthwhile projects.}

Based on the pitch, the concurrency model implemented in the Eventually library is somewhat similar to the event loop model employed by the Node runtime for JavaScript~\cite{NodeJSdocs}. It is more straightforward, less powerful, and smaller - but similar.

The documentation for Eventually is very sparse, which makes it difficult to obtain information outside reading the source code. In short, it is a callback-oriented model with two inbuilt listener types - \textit{timed listener} and \textit{pin listener} - with the option to extend the model by writing additional listeners. When the event is registered, these listeners listen for timing events or pin state events and fire the relevant callback function~\cite{bartlettEventually2022Bartlett}.

One of the issues with a callback-based approach is the concept of callback hell - continuous callbacks inside callbacks, which can become complicated to read. This problem is partially addressed with Promises in Node, but no such construct is available in the Eventually library.

Listing~\ref{lst:eventuallyexample} shows the Eventually implementation of the sample project. It shows the use of both types of built-in listeners and the difficulty of describing everything with listenable events, even in a relatively simple project.

Unfortunately, Eventually has not been updated in more than five years, and the latest changes on the library repository are not released. This lack of release means that to have events in both states of the digital pin, a workaround is required: lines \ref{minted:eventually1} and \ref{minted:eventually2} in Listing 1 create two similar EvtPinListener variables outside the scope of the blinker function. This is done to set the targetValues of the PinListeners in the setup function on lines \ref{minted:eventually3} and \ref{minted:eventually4}.

Eventually will not be used for this project because of the poor documentation, the lack of updates, and the need for the above workaround.


\subsection{Achieving Concurrency with an Operating System}\label{subsec:arduinoos}
Rather than a library, an actual \gls{os} can be installed on the Arduino instead. Several options exist, such as Simba \cite{SimbaOS} and \gls{frtos} \cite{AboutRTOS}. Compared to a library, \glspl{os} are much bigger projects. This quality is typically also reflected in the documentation, and documentation is not a quality to search for when evaluating them. Instead, the overhead (size in particular) is considered.

Only \gls{frtos} is considered for this project, as it is the only \gls{os} mentioned as a solution on the Arduino site, which has a version 1 release.


\subsubsection{FreeRTOS}
Installing a \glspl{os} on a Arduino is the same way as a library is installed - by including them in the code. Unlike the libraries from subsection~\ref{subsec:arduinolibraries}, \gls{frtos} contains additional libraries which may be included together with the \gls{os}. These additional libraries contain the non-core features of the \gls{os}, for example, semaphores, which may be included if needed.


\begin{listing}[htb!]
  \centering
  \begin{minted}[label=FreeRTOS example]{arduino}
    #include <Arduino_FreeRTOS.h>
    #define BUTTON_PIN 12
    void TaskBlink(void *pvParameters){ // This is a task.
      (void) pvParameters;
      pinMode(LED_BUILTIN, OUTPUT);
      for (;;) { // A Task shall never return or exit.
        digitalWrite(LED_BUILTIN, HIGH);
        vTaskDelay( 1000 / portTICK_PERIOD_MS );
        digitalWrite(LED_BUILTIN, LOW);
        vTaskDelay( 1000 / portTICK_PERIOD_MS );
      }
    }
    void TaskDigitalRead(void *pvParameters){ // This is a task.
      (void) pvParameters;
      for (;;) {
        int sensorValue = digitalRead(BUTTON_PIN);
        Serial.println(sensorValue);
        vTaskDelay(1);
      }
    }
    void setup() {
      Serial.begin(9600);
      while (!Serial) {;}
      xTaskCreate(TaskBlink, "Blink", 128, NULL, 2, NULL);
      xTaskCreate(TaskDigitalRead, "DigitalRead", 128, NULL, 1, NULL);
    }
    void loop(){}
\end{minted}
  \caption{Free RTOS implementation of the sample project.}
  \label{lst:freeftosexample}
\end{listing}


\blockcquote{AboutRTOS}{\gls{frtos} is an operating system specifically designed for microcontrollers and microcomputers, such as the Arduino. It has been developed in partnership with the leading chip companies in the world, over more than 18 years, with special emphasis on reliability, accessibility and ease of use.}

From the pitch, \gls{frtos} sounds promising, and the list of available features is much more extensive than Protothreads. Where Protothreads only allow for cooperative scheduling (described explicitly by the programmer), \gls{frtos} has support for cooperative and preemptive scheduling, as well as scheduling, handled implicitly.

Comparing the \gls{frtos} implementation of the sample project from Listing~\ref{lst:freeftosexample} and Listing~\ref{lst:protothreadsexample}, the base implementation of \gls{frtos} resembles Protothreads. Both implementations describe similar tasks and set up the task in the setup() function. Protothreads also schedules the loop function's tasks, but this is nothing major.

Interestingly, YIELD is not used in \gls{frtos}. This detail shows the explicit control of Protothreads, which the programmer has to describe. In \gls{frtos}, the same line contains a delay - this is simply a safety detail and makes no real difference. Additional libraries, such as <task.h>, may be included for \gls{frtos} to access similar explicit control.

Another critical difference is the size and strength of each task. Unlike Protothreads, \gls{frtos} uses a proper stack with context switching, and as a consequence, each stack is much larger. Protothreads use 2 bytes per stack, while \gls{frtos} uses 64 bytes + task stack size~\cite{AboutRTOS}. On the other hand, this allows \gls{frtos} tasks to have local variables. The library size of Protothreads is also a lot smaller than the \gls{frtos} installation.


\begin{figure}[htbp]
  \centering
  \includegraphics[width=0.8\textwidth]{figures/FreeRTOS_Prioritization.png}
  \caption{Scheduling of three tasks with prioritization~\cite{SchedulingRTOS}.}
  \label{fig:freertosprio}
\end{figure}


Because of the scheduler being much more complex, \gls{frtos} also makes prioritization possible. Figure~\ref{fig:freertosprio} illustrates how the scheduler can be instructed to prioritize specific tasks, (in this case, vControlTask, higher than vKeyHandlerTask), which is again prioritized higher than the idle task. This concept is not supported in Protothreads and can only be simulated partially by scheduling a task more frequently than another, which has its own drawbacks.


\subsection{Summary}\label{subsec:concurrencysummary}
\gls{frtos} is more powerful than Protothreads, with a powerful scheduler supporting multiple concurrency models. However, compared to Protothreads, it also has a much larger memory footprint, especially as the amount of tasks in the program increases.

To focus on the design of a programming language aimed at hobbyists - with few surprises, such as surprising memory constraints - the more straightforward Protothreads library is chosen for this project. This choice limits the options available for language constructs related to concurrency in the programming language design but should still be enough for a small language.