\section{Problem statement}\label{sec:problemstatement}

\subsection{What is the problem}\label{sec:whatistheproblem}
The problem we will try to  solve in this project is an extension for hobbyists\todo{Giovanni: ??} working on Arduino, concurrency and real-time systems \todo{Real-time systems has never been mentioned}. The problem is based on hobbyists having trouble implementing concurrency simply and effectively without getting into the nitty-gritty details, that a hobbyist does not want to deal with. It is also based on many tutorials, which have implementations containing bad practices.% Examples of this can be seen here() /and here where they implement things like interrupt which is not a good way to do concurrency.(explain interupts)

This has led us to the problem we want to try and solve, hobbyists have a hard time using concurrency easily on Arduino. We will try to solve this by creating a programming language for hobbyists to easily and correctly implement concurrency on their Arduino devices. Therefore our problem statement is:

How can you create a programming language that would allow a hobby-level user of Arduino to use concurrency easily and correctly concerning the best Arduino concurrency practices?


\subsection{Concurrency for hobbyist} \feedback{Try to polish your argument}
Concurrency is not achieved effortlessly with Arduino, as an additional layer of programming complexity is necessary. This is, of course, a problem for the hobbyist who is not the most experienced programmer, since it makes the problem of getting the Arduino to do what they want even harder. It also increases the time the hobbyist spends implementing a solution that seems relatively intuitive. To help the hobbyist achieve correct and functional concurrency easily and simply, this is what the project will try to do.\todo{Early conclusion and problem statement. Perhaps too early? If not, it is a bit superficial.}


\urgent{unfinished section. We want only one problem statement, and preferably some references to previous choices}
The problem statement goes here. Temporary problem statement (Probably a full language, rather than an extension):

\blockquote{To create an extension on the Arduino programming language to support concurrency in real-time systems. The language should aid hobbyists in writing idiomatic Arduino programs.}

\subsection{Second attempt}

\blockquote{To create a programming language for Arduino, which leverages some concurrency mechanism, to make it simpler or easier for hobbyists to write idiomatic Arduino programs with concurrency.}


\blockquote{To create a programming language for Arduino, which leverages some concurrency mechanism, to make it simpler or easier for hobbyists to write idiomatic Arduino programs with concurrency.}