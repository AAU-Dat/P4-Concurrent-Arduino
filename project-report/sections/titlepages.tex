\pdfbookmark[0]{English title page}{label:titlepage_en}
\aautitlepage{%
  \englishprojectinfo{
    Concurrency in Arduino %title
  }{%
    Design definition and implementation programming languages %theme
  }{%
    Spring Semester 2022 %project period
  }{%
    cs-22-dat-4-03 % project group
  }{%
    %list of group members
    Christoffer Trebbien Jønsson\\
    Daniel Runge Petersen\\
    Gustav Svante Grønkjær Graversen\\
    Jamie Lee Smith Hammer\\
    Lars Emanuel Hansen\\
    Sebastian Aaholm
  }{%
    %list of supervisors
    Giovanni Bacci
  }{%
    1 % number of printed copies
  }{%
    \today % date of completion
  }%
}{%department and address
  \textbf{Electronics and IT}\\
  Aalborg University\\
  \href{http://www.aau.dk}{http://www.aau.dk}
}{% the abstract
  This report documents the design and implementation of Arc. The project's goal was to create a programming language for hobbyists to work with concurrency on an Arduino.

  Our solution to the problem statement is the Arc language, which transpiles into the Arduino language and implements Protothreads to handle concurrency.

  Arc has been succesfully implemented in Java using \acrshort{antlr}4 for the lexer and parser generation, with the scope and type-checking handled by extending \acrshort{antlr}'s BaseVisitor class using the visitor pattern.

  Pending the completion of the described usability test, a final evaluation of Arc as a solution to the problem statement can be made.
}


