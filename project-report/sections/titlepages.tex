\pdfbookmark[0]{English title page}{label:titlepage_en}
\aautitlepage{%
  \englishprojectinfo{
    Concurrency in Arduino %title
  }{%
    Design definition and implementation programming languages %theme
  }{%
    Spring Semester 2022 %project period
  }{%
    cs-22-dat-4-03 % project group
  }{%
    %list of group members
    Christoffer Trebbien Jønsson\\
    Daniel Runge Petersen\\
    Gustav Svante Grønkjær Graversen\\
    Jamie Lee Smith Hammer\\
    Lars Emanuel Hansen\\
    Sebastian Aaholm
  }{%
    %list of supervisors
    Giovanni Bacci
  }{%
    1 % number of printed copies
  }{%
    \today % date of completion
  }%
}{%department and address
  \textbf{Electronics and IT}\\
  Aalborg University\\
  \href{http://www.aau.dk}{http://www.aau.dk}
}{% the abstract
This report will document the design and implementation of Arc.
The project's goal was to create a simple language for hobbyists to work with concurrency on an Arduino. It will look into some of the problems concerning concurrency and Arduino. This leads to the problem statement, "The project aims to create a programming language for hobbyists that satisfies our language criterion and makes it possible to express concurrency on the Arduino easily". Our solution to this is the Arc language, which transpiles into the Arduino language and implements Protothreads to handle concurrency. The compiler has been implemented in Java using \acrshort{antlr}4 for the lexer and parser generation. The scope and type checking is handled by extending \acrshort{antlr}'s BaseVisitor class following the visitor pattern. In conclusion, Arc has implemented concurrency and is a reasonable solution to the problem statement. However, we cannot validate that Arc is easy to learn and a simple language because we did not have time to finish our usability test.
}


