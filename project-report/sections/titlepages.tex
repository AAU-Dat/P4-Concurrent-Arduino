\pdfbookmark[0]{English title page}{label:titlepage_en}
\aautitlepage{%
  \englishprojectinfo{
    Concurrency in Arduino %title
  }{%
    Design definition and implementation programming languages %theme
  }{%
    Spring Semester 2022 %project period
  }{%
    cs-22-dat-4-03 % project group
  }{%
    %list of group members
    Christoffer Trebbien Jønsson\\
    Daniel Runge Petersen\\
    Gustav Svante Grønkjær Graversen\\
    Jamie Lee Smith Hammer\\
    Lars Emanuel Hansen\\
    Sebastian Aaholm
  }{%
    %list of supervisors
    Giovanni Bacci
  }{%
    1 % number of printed copies
  }{%
    \today % date of completion
  }%
}{%department and address
  \textbf{Electronics and IT}\\
  Aalborg University\\
  \href{http://www.aau.dk}{http://www.aau.dk}
}{% the abstract
  This report aims to create a simple language for hobbists, to work with concurrency on an Arduino. Its entry point is what the existing solutions are, and describing what the problems with working concurrently with Arduino are. Which culminates in the problem statement "To create a programming language for the Arduino with a set of concurrency constructs, that leverages the Protothreads library, to concisely
  express a concurrent flow of control.". The Arc language, is designed by means of big and small step semantics, and static scopes. The compiler is written in Java and compiles to C++ and uses a visitor pattern to type and scope check. 
  A usability test was planned, but due to a time constraint, the test was not conducted. Then Arcs language design and more are discussed, to display why certain descisions were made and what could have been done better.
  In conclusion Arc has implemented and features of concurrency and solved the problem statement to a statisfying degree. However the Arc cannot be validated as a easy to learn and simple language because the usability has not been conducted.
   
  \todo[inline]{Abstract is a bit long, consider cutting parts of it}
}


